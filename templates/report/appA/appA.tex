\chapter{Recipe for Cinnamon Balls}
\label{app:cinball}

\begin{table}[h]
\centering
\begin{tabular}{c|c}
\hline
\hline
\textbf{Ingredient}           & \textbf{Amount}  \\
\hline
egg whites                    & 2                  \\                
Castor (superfine) sugar      & 100\,g (4\,oz / $\frac{1}{2}$\,cup) \\
ground almonds       	      & 200\,g ($1/2$\,lb / 2\,cups)      \\ 
cinnamon		      & 1 level tablespoon         \\        
icing (confectioners') sugar  & (approx 5mm deep in a plate or wide bowl)\\    
\hline      
\end{tabular}
\end{table}        

\begin{algorithm}[h]
\begin{algorithmic}[1]
\STATE Beat the egg whites till they form stiff peaks.
\STATE Fold in all the remaining ingredients.
\STATE Form into balls with wetted hands.
\STATE Bake on a greased tray at $170^{\circ}$\,C (Gas Mark 3 / $325^{\circ}$\,F) for 25 minutes, or until just firm to the touch.
\STATE Roll in icing sugar whilst warm.
\STATE Roll in icing sugar when cold.
\end{algorithmic}
\caption[Recipe for Cinnamon Balls.]{The Perfect Cinnamon Balls \citep{Rose04}}
\end{algorithm}

I find it easier to mix the dry ingredients first, before adding them to the egg whites. This ensures a more even mixing.

It is important to bake the balls only as long as directed to ensure that the biscuits remain soft and moist inside. 
It may seem that they are still underdone, but it is important that they are not allowed to dry out.

These amounts make about 15-20 depending on the size of the cinnamon balls. 
I find it best {\em not} to pre--heat the oven otherwise they may burn. 
Also use a clean baking sheet, or one that has only been used for 
cakes/biscuits, to improve taste. Remove them from the baking sheet with a firm twist, or a thin spatula.

It is also possible to replace cinnamon with the contents of a vanilla pod and a tea-spoon of
vanilla essence, to make Vanilla Balls!
