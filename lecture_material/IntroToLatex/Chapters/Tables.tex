\chapter{Tables}
Tables are big business in Latex.  I use these packages to help me.

\begin{verbatim}
\usepackage{multirow}
\usepackage{booktabs}
\usepackage{dcolumn}
\end{verbatim}

Here's an example table

\begin{table}[!hb]
 \centering
	\begin{tabular}{@{}rccc@{~}r@{.}l*3{r@{.}l@{~}r@{.}l}D{.}{\cdot}{2,1}}
	\toprule
	\multicolumn{1}{c}{Fibre Type}
	& N 
	& M 
	&\multicolumn{3}{c}{$L_t$ (nm)}
	&\multicolumn{4}{c}{$R_t$ (nm)}
	&\multicolumn{4}{c}{$d_t$ (nm)}
	&\multicolumn{4}{c}{${\Delta}Z_t$ (nm)}
	&\multicolumn{1}{c}{~}\\
\cmidrule(lr){1-1}
\cmidrule(lr){2-2}
\cmidrule(lr){3-3}
\cmidrule(lr){4-6}
\cmidrule(lr){7-10}
\cmidrule(lr){11-14}
\cmidrule(lr){15-18}
  SS Twisted & 5 & 4 & 313 &\multicolumn{2}{c}{(122)} & 5&7 &(1&6)  & 5&4 &(1&0)  &  150&5 &(63&1) &\multicolumn{1}{c}{~} \\
  \multicolumn{19}{c}{~}\\
  %\cmidrule(lr){3-18}
  &
  &
 	&\multicolumn{3}{c}{$L_s$ (nm)} 
 	&\multicolumn{4}{c}{$R_s$ (nm)}
 	&\multicolumn{4}{c}{$W_s$ (nm)}
 	&\multicolumn{4}{c}{${\Delta}Z_s$ (nm)}
 	&\multicolumn{1}{c}{${\Delta}Z_{e}$ (nm)}\\
\cmidrule(lr){4-6}
\cmidrule(lr){7-10}
\cmidrule(lr){11-14}
\cmidrule(lr){15-18}
\cmidrule(lr){19-19}
  SS Spiral& 32 & 7 &123&(31&6)&11&4&(3&3)&11&1&(2&6)&32&9&(22&2)&26.9\\
  SSB Spiral& 64&18 &106&(22&9)&10&8&(2&7)&9&8&(1&8)&28&2&(8&1)&22.7\\
	%\midrule
  BSS Spiral&37 &19 &110&(44&9)&13&4&(3&0)&12&9&(1&8)&27&1&(9&1)&21.5\\
	%\midrule
  SSSB Spiral&26& 6 &104&(23&9)&13&3&(2&6)&16&4&(3&4)&29&9&(11&1)&22.5\\
	\bottomrule
	\end{tabular}	
	\caption[Basic Fibre Dimensions by TEM]{A funky table from my thesis.}
	\label{tab:BasicXSBFibreDimensionsTEM}
\end{table}

\pagebreak
Here's what the code looks like:
\begin{verbatim}
\begin{table}[!hb]
 \centering
	\begin{tabular}{@{}rccc@{~}r@{.}l*3{r@{.}l@{~}r@{.}l}D{.}{\cdot}{2,1}}
	\toprule
	\multicolumn{1}{c}{Fibre Type}
	& N 
	& M 
	&\multicolumn{3}{c}{$L_t$ (nm)}
	&\multicolumn{4}{c}{$R_t$ (nm)}
	&\multicolumn{4}{c}{$d_t$ (nm)}
	&\multicolumn{4}{c}{${\Delta}Z_t$ (nm)}
	&\multicolumn{1}{c}{~}\\
\cmidrule(lr){1-1}
\cmidrule(lr){2-2}
\cmidrule(lr){3-3}
\cmidrule(lr){4-6}
\cmidrule(lr){7-10}
\cmidrule(lr){11-14}
\cmidrule(lr){15-18}
  SS Twisted & 5 & 4 & 313 &\multicolumn{2}{c}{(122)} & 5&7 &(1&6)  & 5&4 &(1&0)  &  150&5 &(63&1) &\multicolumn{1}{c}{~} \\
  \multicolumn{19}{c}{~}\\
  %\cmidrule(lr){3-18}
  &
  &
 	&\multicolumn{3}{c}{$L_s$ (nm)} 
 	&\multicolumn{4}{c}{$R_s$ (nm)}
 	&\multicolumn{4}{c}{$W_s$ (nm)}
 	&\multicolumn{4}{c}{${\Delta}Z_s$ (nm)}
 	&\multicolumn{1}{c}{${\Delta}Z_{e}$ (nm)}\\
\cmidrule(lr){4-6}
\cmidrule(lr){7-10}
\cmidrule(lr){11-14}
\cmidrule(lr){15-18}
\cmidrule(lr){19-19}
  SS Spiral& 32 & 7 &123&(31&6)&11&4&(3&3)&11&1&(2&6)&32&9&(22&2)&26.9\\
  SSB Spiral& 64&18 &106&(22&9)&10&8&(2&7)&9&8&(1&8)&28&2&(8&1)&22.7\\
	%\midrule
  BSS Spiral&37 &19 &110&(44&9)&13&4&(3&0)&12&9&(1&8)&27&1&(9&1)&21.5\\
	%\midrule
  SSSB Spiral&26& 6 &104&(23&9)&13&3&(2&6)&16&4&(3&4)&29&9&(11&1)&22.5\\
	\bottomrule
	\end{tabular}	
	\caption[Basic Fibre Dimensions by TEM]{A funky table from my thesis.}
	\label{tab:BasicXSBFibreDimensionsTEM}
\end{table}
\end{verbatim}

\pagebreak

\section{Tables Made Easy}
Here is a simple table followed by the code that produced it.

\begin{center}
\begin{tabular}{lcr}
anchovy & banana & carrot \\
dog & apple & fennel \\
goat & strawberry & potato
\end{tabular}

\vspace*{2ex}

\begin{verbatim}
\begin{tabular}{lcr}
anchovy & banana & carrot \\
dog & apple & fennel \\
goat & strawberry & potato
\end{tabular}
\end{verbatim}
\end{center}

\vspace*{2ex}

The tabular environment is a special case of the ''array'' environment for distributing content uniformly across a region of the page.  This ability can be exploited in mathematical equations which we'll see that later on. For now though look at the first line.

\vspace*{2ex}

\begin{verbatim}
\begin{tabular}{lcr}
\end{verbatim}

\vspace*{2ex}

This command tells LaTeX to enter the tabular environment. The letters l, c and r in the curly braces tell LaTeX to create a table with three columns in which the first column is left justified, the second column is centered and the third column is right justified. Lets add a fourth column and this time center justify all the columns.

\begin{center}
\begin{tabular}{cccc}
anchovy & banana & carrot & Johnny\\
dog & apple & fennel & Pete\\
goat & strawberry & potato &
\end{tabular}

\vspace*{2ex}

\begin{verbatim}
\begin{tabular}{cccc}
anchovy & banana & carrot & Johnny\\
dog & apple & fennel & Pete\\
goat & strawberry & potato &
\end{tabular}
\end{verbatim}
\end{center}

\vspace*{2ex}

Each row in the table is a list of items separated by the {\&} symbol. The end of each row is denoted by \textbackslash\textbackslash.  The last row in the table doesn't have a \textbackslash\textbackslash. You do not have to have data between the ampersands but you must have the right number of ampersands to match the number of columns that LaTeX is expecting.

\subsection{Adding Borders To Tables}
\label{sec:vertLines}
Tables should never have vertical lines. No professionally typeset table contains vertical lines. Do not put vertical lines in your tables. That said it is easy to do.

\begin{center}
\begin{tabular}{|c|c|c|c|}
anchovy & banana & carrot & Johnny\\
dog & apple & fennel & Pete\\
goat & strawberry & potato &
\end{tabular}

\vspace*{2ex}

\begin{verbatim}
\begin{tabular}{|c|c|c|c|}
anchovy & banana & carrot & Johnny\\
dog & apple & fennel & Pete\\
goat & strawberry & potato &
\end{tabular}
\end{verbatim}
\end{center}

\vspace*{2ex}

Tables should have neatly headed columns with the heading for each field separated from the data by horizontal lines. The {\textbackslash}toprule, {\textbackslash}cmidrule{} and {\textbackslash}bottomrule commands from the booktabs package are useful for controlling horizontal lines.

\begin{center}
\begin{tabular}{cccc}
\toprule
Ingredient 1 & Ingredient 2 & Ingredient 3 & Source \\
\cmidrule(){1-4}
anchovy & banana & carrot & Johnny\\
dog & apple & fennel & Pete\\
goat & strawberry & potato & \\
\bottomrule
\end{tabular}

\vspace*{2ex}

\begin{verbatim}
\begin{tabular}{cccc}
\toprule
Ingredient 1 & Ingredient 2 & Ingredient 3 & Source \\
\cmidrule(){1-4}
anchovy & banana & carrot & Johnny\\
dog & apple & fennel & Pete\\
goat & strawberry & potato & \\
\bottomrule
\end{tabular}
\end{verbatim}
\end{center}

Note that when using the {\textbackslash}bottomrule command you must add the \textbackslash\textbackslash ~symbol to the last line of data.  The last line of the table is now buried within the {\textbackslash}bottomrule command.

\pagebreak
\subsection{The {\textbackslash}cmidrule Command}
This useful and versatile command takes a bunch of options to control subtleties like only putting lines across some of the columns, or not quite making them cross the full width of the column. The (lr) option trims the left and right ends of the lines off. For example:

\begin{center}
\begin{tabular}{ccccc}
\toprule
Recipe Version & Ingredient 1 & Ingredient 2 & Ingredient 3 & Source \\
\cmidrule(lr){1-1}
\cmidrule(l){2-2}
\cmidrule(){3-3}
\cmidrule(r){4-4}
\cmidrule(lr){5-5}
10.1 & anchovy & banana & carrot & Johnny\\
1.34 & dog & apple & fennel & Pete\\
709.23 & goat & strawberry & potato & \\
\bottomrule
\end{tabular}

\vspace*{2ex}

\begin{verbatim}
\begin{tabular}{ccccc}
\toprule
Recipe Version & Ingredient 1 & Ingredient 2 & Ingredient 3 & Source \\
\cmidrule(lr){1-1}
\cmidrule(l){2-2}
\cmidrule(){3-3}
\cmidrule(r){4-4}
\cmidrule(lr){5-5}
10.1 & anchovy & banana & carrot & Johnny\\
1.34 & dog & apple & fennel & Pete\\
709.23 & goat & strawberry & potato & \\
\bottomrule
\end{tabular}
\end{verbatim}
\end{center}

\pagebreak
\subsection{Aligning Decimal Points Trick One}
Note that the decimal points don't line up in the new column ''Recipe Version'' in the previous section. There are some tricks to deal with this.

\begin{center}
\begin{tabular}{r@{.}lcccc}
\toprule
\multicolumn{2}{c}{Recipe Version} & Ingredient 1 & Ingredient 2 & Ingredient 3 & Source \\
\cmidrule(lr){1-2}
\cmidrule(lr){3-3}
\cmidrule(lr){4-4}
\cmidrule(lr){5-5}
\cmidrule(lr){6-6}
10&1 & anchovy & banana & carrot & Johnny\\
1&34 & dog & apple & fennel & Pete\\
709&23 & goat & strawberry & potato & \\
\bottomrule
\end{tabular}

\vspace*{2ex}

\begin{verbatim}
\begin{tabular}{r@{.}lcccc}
\toprule
\multicolumn{2}{c}{Recipe Version} & Ingredient 1 & Ingredient 2 & Ingredient 3 & Source \\
\cmidrule(lr){1-2}
\cmidrule(lr){3-3}
\cmidrule(lr){4-4}
\cmidrule(lr){5-5}
\cmidrule(lr){6-6}
10&1 & anchovy & banana & carrot & Johnny\\
1&34 & dog & apple & fennel & Pete\\
709&23 & goat & strawberry & potato & \\
\bottomrule
\end{tabular}
\end{verbatim}
\end{center}

My version numbers now use two columns. One is right justified and the other is left justified.  The sequence of symbols @\{.\} tells LaTeX to replace the border between the columns with a period\footnote{Notice that the vertical lines in the tables in section \ref{sec:vertLines} are created by placing the $|$ symbol at the border between the columns.}. If I place the part of the number before the period in the first right justified column and the part of the number after the period in the second left justified column then, with the border between the adjacent table cells delineated by a period, it has the appearance of a decimal number, but all the numbers in the table are now aligned at the decimal point.

I also added the command {\textbackslash}multicolumn\{n\}\{j\}\{text\} to span the heading across both columns. The parameters of the {\textbackslash}multicolumn command are n = number of columns to span, j = justification (l, c or r) and ''text'' is the data to appear in the column. There is a similar {\textbackslash}multirow command for stretching information over several rows in a table which works in a similar way. Look it up.
\pagebreak
\subsection{Aligning Decimal Points Trick Two}
The previous trick for aligning decimal points does not centre the number under the heading with looks rubbish. You could have a shorter heading to reduce this effect or you can use the DColumn package which is a better more general solution for this problem. (I taught the first trick to you so you came across the multicolumn command and learned more about the subtleties of how tables work, and to get you used to thinking in Tablese.).

\begin{center}
\begin{tabular}{D{.}{\cdot}{4,4}cccc}
\toprule
Recipe Version & Ingredient 1 & Ingredient 2 & Ingredient 3 & Source \\
\cmidrule(lr){1-1}
\cmidrule(lr){2-2}
\cmidrule(lr){3-3}
\cmidrule(lr){4-4}
\cmidrule(lr){5-5}
10.1 & anchovy & banana & carrot & Johnny\\
1.34 & dog & apple & fennel & Pete\\
709.23 & goat & strawberry & potato & \\
\bottomrule
\end{tabular}

\vspace*{2ex}

\begin{verbatim}
\begin{tabular}{D{.}{\cdot}{4,4}cccc}
\toprule
Recipe Version & Ingredient 1 & Ingredient 2 & Ingredient 3 & Source \\
\cmidrule(lr){1-1}
\cmidrule(lr){2-2}
\cmidrule(lr){3-3}
\cmidrule(lr){4-4}
\cmidrule(lr){5-5}
10.1 & anchovy & banana & carrot & Johnny\\
1.34 & dog & apple & fennel & Pete\\
709.23 & goat & strawberry & potato & \\
\bottomrule
\end{tabular}
\end{verbatim}
\end{center}

The DColumn package defines a new type of column which can be invoked placing a capital D in the {\textbackslash}tabular command which defines the table. If defined using a D, then the column is placed in mathmode. D takes four parameters: D\{a\}\{b\}\{c,d\} where a is the symbol which is to be aligned, b is the symbol with which to replace the aligning character, and {c,d} must be integers which indicate LaTeX should have up to c white space characters before the aligning character and d afterwards, thereby defining the position of the number within the column. The command {\textbackslash}cdot prints a special type of mathmode symbol which is a dot that is vertically shifted and larger than a normal period e.g.: $\cdot$. 

\pagebreak
\subsection{Final Table Trick}
DColumn forces the column to be in mathmode which is why the heading ''Recipe Version'' was typeset in italics in the previous section. Indeed LaTeX tried to interpret the heading of the table as a number to be aligned.  We can over ride this behaviour by using the {\textbackslash}multicolumn command to locally impose a different type of justification and temporarily disable mathmode as follows:

\begin{center}
\begin{tabular}{D{.}{\cdot}{4,4}cccc}
\toprule
\multicolumn{1}{c}{Recipe Version}& Ingredient 1 & Ingredient 2 & Ingredient 3 & Source \\
\cmidrule(lr){1-1}
\cmidrule(lr){2-2}
\cmidrule(lr){3-3}
\cmidrule(lr){4-4}
\cmidrule(lr){5-5}
10.1 & anchovy & banana & carrot & Johnny\\
1.34 & dog & apple & fennel & Pete\\
709.23 & goat & strawberry & potato & \\
\bottomrule
\end{tabular}

\vspace*{2ex}

\begin{verbatim}
\begin{tabular}{D{.}{\cdot}{4,4}cccc}
\toprule
\multicolumn{1}{c}{Recipe Version} & Ingredient 1 & Ingredient 2 & Ingredient 3 & Source \\
\cmidrule(lr){1-1}
\cmidrule(lr){2-2}
\cmidrule(lr){3-3}
\cmidrule(lr){4-4}
\cmidrule(lr){5-5}
10.1 & anchovy & banana & carrot & Johnny\\
1.34 & dog & apple & fennel & Pete\\
709.23 & goat & strawberry & potato & \\
\bottomrule
\end{tabular}
\end{verbatim}
\end{center}


\pagebreak
\section{Numbering Tables}
To tell LaTeX to assign a number to a table and add it to the list of tables you must use the {\textbackslash}begin\{table\} command to tell LaTeX to create a table environment as follows:

\begin{table}[!bh]
\centering
\begin{tabular}{D{.}{\cdot}{4,4}cccc}
\toprule
\multicolumn{1}{c}{Recipe Version}& Ingredient 1 & Ingredient 2 & Ingredient 3 & Source \\
\cmidrule(lr){1-1}
\cmidrule(lr){2-2}
\cmidrule(lr){3-3}
\cmidrule(lr){4-4}
\cmidrule(lr){5-5}
10.1 & anchovy & banana & carrot & Johnny\\
1.34 & dog & apple & fennel & Pete\\
709.23 & goat & strawberry & potato & \\
\bottomrule
\end{tabular}
\caption[Table of Banned Recipes]{Recipes that ought to be banned.}
\label{tab:Recipes}
\end{table}
\begin{verbatim}
\begin{table}[!bh]
\centering
\begin{tabular}{D{.}{\cdot}{4,4}cccc}
\toprule
\multicolumn{1}{c}{Recipe Version}& Ingredient 1 & Ingredient 2 & Ingredient 3 & Source \\
\cmidrule(lr){1-1}
\cmidrule(lr){2-2}
\cmidrule(lr){3-3}
\cmidrule(lr){4-4}
\cmidrule(lr){5-5}
10.1 & anchovy & banana & carrot & Johnny\\
1.34 & dog & apple & fennel & Pete\\
709.23 & goat & strawberry & potato & \\
\bottomrule
\end{tabular}
\caption[Table of Banned Recipes]{Recipes that ought to be banned.}
\label{tab:Recipes}
\end{table}
\end{verbatim}

In the table environment the {\textbackslash}caption[text1]\{text2\} command adds a caption, where text1 appears in the list of tables at the beginning of the document and text2 is the local caption.  The label command creates a label with which to reference the table e.g. Table \ref{tab:Recipes} is a table of recipes that have been made up to illustrate how to use tables in LaTeX. We also use the {\textbackslash}centering command to center the table and caption within the table environment. We could also use the {\textbackslash}begin\{center\} and {\textbackslash}end\{center\} commands.