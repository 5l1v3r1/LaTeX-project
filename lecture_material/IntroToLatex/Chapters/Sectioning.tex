\chapter{The Chapter on Making Chapters}
\label{sec:Chapters}

Right. LaTeX forces you to structure your document. There a series of simple commands for achieving this:

\vspace*{2ex}

{\textbackslash}chapter\{chapterName\}

{\textbackslash}section\{sectionName\}

{\textbackslash}subsection\{subsectionName\}

{\textbackslash}subsubsection\{subsubsectionName\}

\vspace*{2ex}

Easy enough. At the start of each chapter or section just issue one of these commands to name the section and LaTeX will present the chapter heading in the right style of font, which will be the same style of all the other heading names at that level of hierarchy throughout the document. In addition LaTeX assigns a number to that section. For example, this chapter was created using the command:

\vspace*{2ex}

{\textbackslash}chapter\{The Chapter on Making Chapters\} 

\vspace*{2ex}

LaTeX assigns the correct chapter number to each chapter in turn and then puts it in the table of contents, as you can see by looking at the table of contents! Easy.

\pagebreak
\section{The Section About Sections}
\label{sec:Section}
I think you're getting the hang of this. This section was created using the command {\textbackslash}section\{The Section About Sections\}. It appears in the table of contents as section \ref{sec:Section}.

\subsection{The Sub-Section About Sub-Sections}
\label{sec:SubSection}
Now you're really getting the hang of this. This subsection was created using the command {\textbackslash}subsection\{The Sub-Section About Sub-Sections\}. 

\subsubsection{The Sub-Sub-Section About Sub-Sub-Sections}
\label{sec:SubSubSection}
Now you're really getting the hang of this. but.. caught you out! This subsubsection doesn't have a number!  Ha Ha! It was created using the command {\textbackslash}subsubsection\{The Sub-Sub-Section About Sub-Sub-Sections\} in the same vein as all the other subsections/chapters etc. Here we have stopped the depth of the section numbering at the second level with the command:

\vspace*{2ex}

{\textbackslash}setcounter\{secnumdepth\}\{2\}

\vspace*{2ex}

Easy. WYSIWYM.  LaTeX does what you tell it to do. This is both a blessing and a curse.

\pagebreak
\section*{The Section About Sections That Don't Appear in the Table of Contents}
\label{sec:InvisibleSection}
Ok. Now we're throwing a spanner in the works. This section was created using the command:

\vspace*{2ex}

{\textbackslash}section*\{The Section About Sections That Don't Appear in the Table of Contents\}. 

\vspace*{2ex}

Notice how LaTeX has not assigned a number to the section and it doesn't appear in the table of contents. The effect of the * is to suppress the inclusion of a chapter, section or subsection in the automatic numbering. Easy. You can do this at any level. This is useful for things like prefaces, tables of contents or acknowledgements which you may or may not wish to have an entry in the main contents table. Up to you. It's your thesis. Don't just copy me.

\section{A Small Point About Numbering}

In this section notice how the numbering starts from where it left off before we suppressed the numbering on the previous section. Easy. WYSIWYM.
\pagebreak
\section{The Section About Internal Referencing}
\label{sec:InternalReferencing}

There are two related commands:

\vspace*{2ex}

{\textbackslash}label\{labelName\} 

{\textbackslash}ref\{labelName\}

\vspace*{2ex}


The {\textbackslash}label\{\} command allows you to create a label in a particular environment\footnote{That's right. Chapters, Sections and sub-sections etc are environments!}. The label won't appear in the final document. It's just a label which makes it easy to refer back to any particular environment elsewhere in the document. The {\textbackslash}ref\{\} command enables you to insert a reference anywhere in the document to any label in the document.  For example this is section \ref{sec:InternalReferencing}. The names you use in a label can be anything you like but musn't contain whitespace or special characters. I used the two commands:

\vspace*{2ex}

{\textbackslash}label\{sec:InternalReferencing\}

{\textbackslash}ref\{sec:InternalReferencing\}.

\vspace*{2ex}

The astute among you will realise that LaTeX has to read the document several times. Once to find the labels and then again to populate the references with the correct numbers. So you have to compile a latex document twice to get the referencing right. If there is a missing label or you refer to something that doesn't exist then latex inserts a convenient ? at that point. So hunting for queries is useful way of finding broken references. Latex issues warnings when it finds broken references.

As you create a document you will find yourself putting labels in all over the place so choose a sensible naming convention to help you remember the label names.

Each type of environment (equations, figures, tables, sections etc) has its own independent numbering system.  So when you choose your label name it's a good idea to have an identifier for that type of environment.  I have my own convention for label names which I use to help me remember references. E.g.   sec:SectionAboutCats. Eqn:EquationAboutCats,  Fig:FigureAboutCats and so on. This means you can differentiate between referring to the section or the figure more easily, even though they are about the same thing.

So Internal Referencing is a doddle. Easy!