\chapter{Overview}
\label{chap:overview}
\section{LaTeX: The answer to everything.}
LaTeX is typsetting program that takes an input file of marked up text and creates a beautifully crafted output file that can be easily printed (e.g. a pdf). 

The typsetting rules applied by LaTeX are targeted mainly at aesthetics and those rules have been honed by professional typesetters since the time of Caxton! 

The precursor to LaTeX (TeX) was written by a professional typesetter.

\subsection{Advantages}

The advantages of this method are manifold:

\begin{itemize}

\item Use any text editor to view the source document.
\item More time can be spent working on the content and not worrying about how text and figures will appear.
\item LaTeX uses consistent rules throughout a document
\item LaTeX sorts out basic typesetting automatically. 
\item Changes can be introduced globally with very little effort
\item Document structure is explicit
\item Documents can be professionally typeset and look great
\item You are forced to structure your documents correctly.
\item Mathematical equations, like $E=mc^2$ or $\imath\hbar\frac{\partial}{\partial t}\Phi (x, t) = \hat{H}\Phi (x, t)$ can be produced almost as fast as typing (if you know the commands!).
\end{itemize}

\subsection{Disadvantages}
The disadvantages of this method are also manifold:
\begin{itemize}
\item You don't see the output as you go.
\item Steep learning curve.
\item Documents are harder to edit by a second author (unless they are adept at LaTeX too). This can be mitigated using version control, which makes group authoring processes superior to word.
\item The program never works quite the way you want it to and learning how to influence it can be problematic and subtle.
\item Although intended to save work the principle of 'conservation of work' means that you simply transform problems associated with WYSISYG approaches to problems associated with WYSIWYM approaches!
\item You can go blind trying to determine the difference between wiggly and smooth brackets if your editors font isn't large enough.
\end{itemize}

\subsection{On balance?}
If it's so rubbish, why use LaTeX?

\begin{itemize}
\item Large documents are much more easily handled. 
\item In general it is quicker to debug a LaTeX document than typeset an entire thesis manually and consistently.
\item Technical information such as tables, equations and figures are integrated much more smoothly than with word.
\end{itemize}

\section{How does it work?}

The raw text is interspersed with commands, preceded by a \textbackslash, which tell LaTeX what to do with the text. For example you can \textbf{make it bold}, \emph{italic}, or \underline{underlined} with the commands {\textbackslash}textbf\{\}, {\textbackslash}emph\{\} or {\textbackslash}underlined\{\}. More on commands in a moment...


