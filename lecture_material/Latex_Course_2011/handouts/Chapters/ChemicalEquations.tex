\chapter{Chemical Equations}
The mhchem package means you can do basic stuff very easily using {\textbackslash}ce\{\}. For example:

\vspace{2ex}
\begin{verbatim}
\ce{CO2 + C -> 2CO}
\ce{CO2 + C <- 2CO}
\ce{CO2 + C <=> 2CO}
\ce{A-B=C#D\sbond E\dbond F\tbond G}
\end{verbatim}
\vspace{2ex}
\begin{center}
\ce{CO2 + C -> 2CO}

\ce{CO2 + C <- 2CO}

\ce{CO2 + C <=> 2CO}

\ce{A-B=C#D\sbond E\dbond F\tbond G}
\end{center}

\vspace{2ex}
You can also use math mode within chemical equations.
\vspace{2ex}

\begin{center}
\ce{$x\,$ Na(NH4)HPO4 ->[\Delta] (NaPO3)_{$x$} + $x\,$ NH3 ^ + $x\,$ H2O}
\end{center}

\begin{verbatim}
\ce{$x\,$ Na(NH4)HPO4 ->[\Delta](NaPO3)_{$x$} + $x\,$ NH3 ^ + $x\,$ H2O}
\end{verbatim}

And you can number chemical reactions as well by using the math mode equation environment.

\begin{equation}
\ce{CO2 + C <=> 2CO}
\end{equation}

\begin{verbatim}
\begin{equation}
\ce{CO2 + C <=> 2CO}
\end{equation}
\end{verbatim}

However, this will number mathematical and chemical equations using the same number system. There is a discussion in the mhchem pdf which shows you how to number chemical and mathematical equations independently. It's a bit complex and involves you creating your own type of environment... bit beyond the intro level of this course...