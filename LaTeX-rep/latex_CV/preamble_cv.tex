% Premable by Ted Pavlic and modified by Ignas Anikevicius to suit my needs

%%%%%%%%%%%%%%%%%%%%%%%%%%%% Document Setup %%%%%%%%%%%%%%%%%%%%%%%%%%%%
\documentclass[10pt]{article}
% This is a helpful package that puts math inside length specifications
\usepackage{calc}

% Simpler bibsection for CV sections
% (thanks to natbib for inspiration)
\makeatletter
\newlength{\bibhang}
\setlength{\bibhang}{1em}
\newlength{\bibsep}
 {\@listi \global\bibsep\itemsep \global\advance\bibsep by\parsep}
\newenvironment{bibsection}%
        {\vspace{-\baselineskip}\begin{list}{}{%
       \setlength{\leftmargin}{\bibhang}%
       \setlength{\itemindent}{-\leftmargin}%
       \setlength{\itemsep}{\bibsep}%
       \setlength{\parsep}{\z@}%
        \setlength{\partopsep}{0pt}%
        \setlength{\topsep}{0pt}}}
        {\end{list}\vspace{-.6\baselineskip}}
\makeatother

% Layout: Puts the section titles on left side of page
\reversemarginpar

%% Use these lines for letter-sized paper
%\usepackage[paper=letterpaper,
%            %includefoot, % Uncomment to put page number above margin
%            marginparwidth=1.2in,     % Length of section titles
%            marginparsep=.05in,       % Space between titles and text
%            margin=1in,               % 1 inch margins
%            includemp]{geometry}

%% Use these lines for A4-sized paper
\usepackage[paper=a4paper,
            %includefoot, % Uncomment to put page number above margin
            marginparwidth=30.5mm,    % Length of section titles
            marginparsep=1.5mm,       % Space between titles and text
            margin=25mm,              % 25mm margins
            includemp]{geometry}

%% More layout: Get rid of indenting throughout entire document
\setlength{\parindent}{0in}

%% This gives us fun enumeration environments. compactitem will be nice.
\usepackage{paralist}

%% Reference the last page in the page number
%
% NOTE: comment the +LP line and uncomment the -LP line to have page
%       numbers without the ``of ##'' last page reference)
%
% NOTE: uncomment the \pagestyle{empty} line to get rid of all page
%       numbers (make sure includefoot is commented out above)
%
\usepackage{fancyhdr,lastpage}
\pagestyle{fancy}
%\pagestyle{empty}      % Uncomment this to get rid of page numbers
\fancyhf{}\renewcommand{\headrulewidth}{0pt}
\fancyfootoffset{\marginparsep+\marginparwidth}
\newlength{\footpageshift}
\setlength{\footpageshift}
          {0.5\textwidth+0.5\marginparsep+0.5\marginparwidth-2in}
\lfoot{\hspace{\footpageshift}%
       \parbox{4in}{\, \hfill %
                    \arabic{page} of \protect\pageref*{LastPage} % +LP
%                    \arabic{page}                               % -LP
                    \hfill \,}}

% Finally, give us PDF bookmarks
\usepackage{color,hyperref}
\definecolor{darkblue}{rgb}{0.0,0.0,0.3}
\hypersetup{colorlinks,breaklinks,
            linkcolor=darkblue,urlcolor=darkblue,
            anchorcolor=darkblue,citecolor=darkblue}

%%%%%%%%%%%%%%%%%%%%%%%% End Document Setup %%%%%%%%%%%%%%%%%%%%%%%%%%%%


%%%%%%%%%%%%%%%%%%%%%%%%%%% Helper Commands %%%%%%%%%%%%%%%%%%%%%%%%%%%%

% The title (name) with a horizontal rule under it
% (optional argument typesets an object right-justified across from name
%  as well)
%
% Usage: \makeheading{name}
%        OR
%        \makeheading[right_object]{name}
%
% Place at top of document. It should be the first thing.
% If ``right_object'' is provided in the square-braced optional
% argument, it will be right justified on the same line as ``name'' at
% the top of the CV. For example:
%
%       \makeheading[\emph{Curriculum vitae}]{Your Name}
%
% will put an emphasized ``Curriculum vitae'' at the top of the document
% as a title. Likewise, a picture could be included:
%
%   \makeheading[\includegraphics[height=1.5in]{my_picutre}]{Your Name}
%
% the picture will be flush right across from the name.
\newcommand{\makeheading}[2][]%
        {\hspace*{-\marginparsep minus \marginparwidth}%
         \begin{minipage}[t]{\textwidth+\marginparwidth+\marginparsep}%
             {\large \bfseries #2 \hfill #1}\\[-0.15\baselineskip]%
                 \rule{\columnwidth}{1pt}%
         \end{minipage}}

% The section headings
%
% Usage: \section{section name}
%
% Follow this section IMMEDIATELY with the first line of the section
% text. Do not put whitespace in between. That is, do this:
%
%       \section{My Information}
%       Here is my information.
%
% and NOT this:
%
%       \section{My Information}
%
%       Here is my information.
%
% Otherwise the top of the section header will not line up with the top
% of the section. Of course, using a single comment character (%) on
% empty lines allows for the function of the first example with the
% readability of the second example.
\renewcommand{\section}[2]%
        {\pagebreak[3]\vspace{1.3\baselineskip}%
         \phantomsection\addcontentsline{toc}{section}{#1}%
         \hspace{0in}%
         \marginpar{
         \raggedright \scshape #1}#2}

% An itemize-style list with lots of space between items
\newenvironment{outerlist}[1][\enskip\textbullet]%
        {\begin{itemize}[#1]}{\end{itemize}%
         \vspace{-.6\baselineskip}}

% An environment IDENTICAL to outerlist that has better pre-list spacing
% when used as the first thing in a \section
\newenvironment{lonelist}[1][\enskip\textbullet]%
        {\vspace{-\baselineskip}\begin{list}{#1}{%
        \setlength{\partopsep}{0pt}%
        \setlength{\topsep}{0pt}}}
        {\end{list}\vspace{-.6\baselineskip}}

% An itemize-style list with little space between items
\newenvironment{innerlist}[1][\enskip\textbullet]%
        {\begin{compactitem}[#1]}{\end{compactitem}}

% An environment IDENTICAL to innerlist that has better pre-list spacing
% when used as the first thing in a \section
\newenvironment{loneinnerlist}[1][\enskip\textbullet]%
        {\vspace{-\baselineskip}\begin{compactitem}[#1]}
        {\end{compactitem}\vspace{-.6\baselineskip}}

% To add some paragraph space between lines.
% This also tells LaTeX to preferably break a page on one of these gaps
% if there is a needed pagebreak nearby.
\newcommand{\blankline}{\quad\pagebreak[3]}
\newcommand{\halfblankline}{\quad\vspace{-0.5\baselineskip}\pagebreak[3]}

% Uses hyperref to link DOI
\newcommand\doilink[1]{\href{http://dx.doi.org/#1}{#1}}
\newcommand\doi[1]{doi:\doilink{#1}}

% For \url{SOME_URL}, links SOME_URL to the url SOME_URL
\providecommand*\url[1]{\href{#1}{#1}}
% Same as above, but pretty-prints SOME_URL in teletype fixed-width font
\renewcommand*\url[1]{\href{#1}{\texttt{#1}}}

% For \email{ADDRESS}, links ADDRESS to the url mailto:ADDRESS
\providecommand*\email[1]{\href{mailto:#1}{#1}}
% Same as above, but pretty-prints ADDRESS in teletype fixed-width font
%\renewcommand*\email[1]{\href{mailto:#1}{\texttt{#1}}}

%\providecommand\BibTeX{{\rm B\kern-.05em{\sc i\kern-.025em b}\kern-.08em
%    T\kern-.1667em\lower.7ex\hbox{E}\kern-.125emX}}
%\providecommand\BibTeX{{\rm B\kern-.05em{\sc i\kern-.025em b}\kern-.08em
%    \TeX}}
\providecommand\BibTeX{{B\kern-.05em{\sc i\kern-.025em b}\kern-.08em
    \TeX}}
\providecommand\Matlab{\textsc{Matlab}}

%%%%%%%%%%%%%%%%%%%%%%%% End Helper Commands %%%%%%%%%%%%%%%%%%%%%%%%%%%

%%%%%%%%%%%%%%%%%%%%%%%%% Begin CV Document %%%%%%%%%%%%%%%%%%%%%%%%%%%%

% vim: textwidth=80:spell:spelllang=en_gb
