%        File: setup-alt.tex
%     Created: Tue Jul 12 11:00  2011 B
% Last Change: Tue Jul 12 11:00  2011 B
%
\documentclass[a4paper]{article}
\usepackage[T1]{fontenc}
\usepackage{lmodern}
\usepackage{times}

\usepackage{alltt}
\usepackage{verbatim}

\usepackage[]{hyperref}

% Title and similar stuff
\title{How to set up your computer to start using LaTeX}
\author{Ignas Anikevicius}

\begin{document}

\maketitle

\section{Introduction}

Thank you for your interest in \LaTeX typesetting system and this article
will help you to get you ready for starting to use \LaTeX on your
computer.

Although I would like to write a continuous text on how to install
everything on different kinds of OSes (Operating Systems), I believe, that it
is not necessary to duplicate any content, if it can be found in a better
shape elsewhere. Therefore, I suggest you reading chapters of the book called
\LaTeX hosted on the website called wikibooks.org. You can find an on-line
version of the
\href{https://secure.wikimedia.org/wikibooks/en/wiki/LaTeX}{book} or the
\href{http://upload.wikimedia.org/wikipedia/commons/2/2d/LaTeX.pdf}{PDF}
version of it, which I think is much more suitable for reading or printing.

The list of the needed software is already there and if somebody feels very
comfortable with his system, no specific directions should be necessary for
them.

\section{Software from the Department of Chemistry}

Computer Office is already providing images for deploying the whole OS and
necessary software for Chemistry Department members. As far as I was informed,
there are images for Linux and Windows systems. For Macs, there might be
customized installers as well available on \href{http://www.google.co.uk}{this
page}.

\section{\LaTeX distribution installation}

You need either of these:
\begin{itemize}
    \item ``\TeX Live'' \LaTeX distribution which is available for
        Linux/Mac/Windows, but should be preferred on Linux machines.
    \item ``MacTeX'' \LaTeX distribution which is available for Mac machines
        only and should be the preferred option on these machines.
    \item ``MikTeX'' \LaTeX distribution which is available for Windows machines
        only and should be the preferred option on these machines.
\end{itemize}

\subsection{Notes for Linux users}

Use your Linux Distribution package manager whenever you can and install ``\TeX
Live'' only from there. These are the terminal commands how to get everything
necessary on various Linux distributions:

\begin{description}
    \item[.deb based] 
        This applies for all distributions which have their packages as files
        with '.deb' extensions. These mainly include all Debian and Ubuntu
        variants and derivatives. Issue these commands in terminal as root and
        you are good to go.
        % FIXME
        \begin{verbatim}
        # apt-get install texlive-most
        \end{verbatim}

    \item[.rpm based]
        This applies for all distributions which have their packages as files
        with '.rpm' extensions. These mainly include OpenSuSe, Fedora, CentOS,
        RedHat and others. Issue these commands in terminal as root and you are
        good to go.
        % FIXME
        \begin{verbatim}
        # yum install texlive-most
        \end{verbatim}

    \item[ArchLinux and derivatives]
        This applies for ArchLinux and Chackra distributions. Both use pacman as
        their packages manager, so the following commands executed as
        root user will suffice.
        % FIXME
        % Need to finish, elaborate on installation
        \begin{verbatim}
        # pacman -S texlive-most
        \end{verbatim}
    \item[Gentoo and derivatives]
        This applies for Gentoo, Funtoo, Sabayon distributions. They all use
        portage as their packages manager, so the following commands executed as
        root user will suffice.
        % FIXME
        % Need to finish, elaborate on installation
        \begin{verbatim}
        # emerge -av texlive
        \end{verbatim}
    \item[Others]
        Go to the project Wikipedias and find what they suggest doing.
\end{description}

\subsection{Notes for Mac users}

For easier experience, just install the full MacTeX installation which can be
found on the following \href{http://www.tug.org/mactex/}{website}.
\footnote{The URL for the website is \url{http://www.tug.org/mactex/}}

\subsection{Notes for Windows users}

For easier experience, download MiKTeX installation files from
\href{http://miktex.org/2.9/setup}{their website}.
\footnote{The URL for the website is \url{http://miktex.org/2.9/setup}}
There are mainly 2 wise options to select:
\begin{description}
    \item[Install everything] Although this might be very convenient as
        one will not have to worry about missing packages, but it takes space.
        On the other hand, slightly more than 1GB of occupied space on modern
        computers will not make a difference.
    \item[Install a base system] This is the alternative, which would take less
        space. What is more, one can select an option where necessary packages
        could be installed on the fly without any user intervention.
\end{description}

\section{Editing a .tex file}

Mainly there are two choices:
\begin{itemize}
    \item IDE (Integrated Development Environment)
    \item a simple text editor
\end{itemize}

While IDEs
\footnote{an example of an IDE for HTML would be
Dreamweaver\textsuperscript{\copyright}. Note, that I am not affiliated with
Adobe\textsuperscript{\copyright} or any of its partners and I do not have
claims towards its trademarks or copyrighted IP.}
generally will provide a user with much more integrated environment,
this does not necessarily mean, that producing \LaTeX documents with an IDE is
generally faster. There are many very powerful text editors, which might have a
steep learning curve, but once mastered, they are very fast. What is more, some
text editors might be better in some tasks than other, so there is no such thing
as ``the best'' IDE or text editor for \LaTeX.

\subsection{Cross-platform software}

The software which is most worth mentioning is listed bellow:
\begin{description}
    \item[VIM \& Emacs] This editor is the best in my opinion. It is very fast,
        lightweight and it can be customized a lot. Although it has a steep
        learning curve, it is very rewarding afterwards and reading any of books
        on VIM would help a lot. Emacs is also good, and many argue that it is
        better than VIM. This has much to do with so-called editor wars.
        \footnote{Editor wars on Wikipedia:
        \url{http://en.wikipedia.org/wiki/Editor_war}}

        Whichever you choose, you will find that they both have extensible
        resources on the internet and very good plug-ins for \LaTeX document
        production.
    \item[LyX] {\bfseries DO NOT USE THIS WORD PROCESSOR!!!} Although it might
        be very tempting to use something similar to Word or LibreOffice Writer
        and achieve \LaTeX typeset document quality, you will often get into
        trouble if you choose this option. LyX might be very good as a reference
        tool as one can search for \LaTeX commands how to do certain things, it
        is not useful in anything else if one wants to USE \LaTeX .
\end{description}

\subsection{Linux software}

\begin{description}
    \item[Kile] This is an IDE for Linux.
\end{description}

\subsection{Mac OS X software}

\begin{description}
    \item[TeXShop] This is probably the best IDE after the aforementioned
        editors for processing \LaTeX documents.
\end{description}

\subsection{Windows software}

\begin{description}
    \item[TeXnicCenter] This is probably the best tool for \LaTeX in this OS
        after the aforementioned editors. Note, that this is an IDE and not a
        text editor.
\end{description}

\section{Bibliography management software}

\begin{itemize}
    \item Jabref
    \item Bibdesk
    \item Vim and Emacs
\end{itemize}

\section{PDF viewers}

Good PDF viewers are different across different platforms. I believe, that you
might say, that Adobe's PDF viewer is very good, but the truth is that it is
very slow and not as stable as others.

A much better alternative might look \textbf{Foxit} PDF reader, which is
available for both Linux and Windows operating systems. However, I found that
this is not as good as others.

\subsection{On Linux}

Linux users have a huge variety of PDF viewers to select from. 

\subsection{On Mac}

The best choices seem to be viewers \textbf{Preview} and \textbf{Skim} as both
are relatively light and provide a good number of features.

\subsection{On Windows}

The best choice would be a \textbf{Sumatra} PDF viewer. Other alternatives need
to be bought or they are half-baked.

\section{Other useful software \& links}

\end{document}

% Editor configuration
% vim: tw=80:spell:spelllang=en_gb
