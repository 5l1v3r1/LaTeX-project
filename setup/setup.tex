%        File: setup.tex
%     Created: Tue Jul 12 11:00  2011 B
% Last Change: Tue Jul 12 11:00  2011 B
%
\documentclass[a4paper]{article}
\usepackage[T1]{fontenc}
\usepackage{lmodern}
\usepackage{times}

\usepackage{alltt}
\usepackage{verbatim}

\usepackage[]{hyperref}

% Title and similar stuff
\title{How to set up your computer to start using LaTeX}
\author{Ignas Anikevicius}

\begin{document}

\maketitle

\section{Introduction}

Thank you for your interest in \LaTeX typesetting system and this article
will help you to get you ready for starting to use \LaTeX on your
computer.

Although I would like to write a continuous text on how to install
everything on different kinds of OSes (Operating Systems), I believe, that it
is not necessary to duplicate any content, if it can be found in a better
shape elsewhere. Therefore, I suggest you reading chapters of the book called
\LaTeX hosted on the website called wikibooks.org. You can find an on-line
version of the
\href{https://secure.wikimedia.org/wikibooks/en/wiki/LaTeX}{book} or the
\href{http://upload.wikimedia.org/wikipedia/commons/2/2d/LaTeX.pdf}{PDF}
version of it, which I think is much more suitable for reading or printing.

The list of the needed software is already there and if somebody feels very
comfortable with his system, no specific directions should be necessary for
them.

\section{Software from the Department of Chemistry}

Computer Office is already providing images for deploying the whole OS and
necessary software for Chemistry Department members. As far as I was informed,
there are images for Linux and Windows systems. For Macs, there are easy
installers already available on \href{http://www.google.co.uk}{this page}.

\section{Linux machines}

\subsection{Introduction}

There are various Linux Distributions, which are different in their philosophy
and usually they have different base software. For our purposes the most
important differences are those in how the packages are distributed and
installed. If one feels comfortable with the package manager of his
distribution, then the setup of the computer should be a very strait forward
thing to do.

\subsection{Debian family or '.deb' distributions}

All Debian based distributions share a common package managing system and they
have so-called .deb binary packages being distributed. As for list of the
distributions which are based on Debian it is very lengthy, mainly because
Ubuntu is based on Debian and all Ubuntu derivatives should fall under this
category as well.

% How to install the needed packages?

\subsection{Redhat family or '.rpm' distributions}

Here all distributions using '.rpm' packages would fall under this category.
This includes Redhat, CentOS, Fedora, OpenSuSe and similar.

\subsection{ArchLinux, Chakra}

ArchLinux is a community based project and it usually provides a 'bleeding-edge'
packages and sometimes things break. So once you have your \TeX setup up and
running do not update the system if you need a stable system. However, ArchLinux
probably provides the easiest and fastest way to install all the needed
packages. Just execute these commands as root:

\begin{verbatim}
    pacman -S texlive-most
\end{verbatim}

\subsection{Gentoo, Funtoo, Sabayon}

\section{Mac OS X}

One can follow the links on the and get the '.dmg' files from the internet or
use the aforementioned Departament installers. The recommended software
includes:

\begin{itemize}
    \item Full installation of MacTeX distribution
    \item Editors:
        \begin{itemize}
            \item For people who like well IDEs (Integrated Development
                Environments), TeXShop is a really great choice
            \item For people not being afraid of a steep learning curve, but
                very rewarding results, Vim/Emacs should be their main choice
        \end{itemize}
    \item Bibliography managers:
        \begin{itemize}
            \item BibDesk could serve most of the people very well
            \item Another approach would be to use the same Vim/Emacs setup as
                both editors have a lot plug-ins to deal with various things
        \end{itemize}
    \item PDF viewers:
        \begin{itemize}
            \item If you do not care about loading times and stability of the
                software but prefer a really elaborate feature-set, then choose
                Adobe's products.
            \item If you want a lighter alternative, then you can use Preview.
                Skim might be also worth looking at.
        \end{itemize}
    \item Graphics post processing tools:
        \begin{itemize}
            \item Adobe Illustrator will definitely fulfil your requirements,
                but it is really heavy.
            \item A lighter but still capable of producing good quality graphics
                could be open-source Inkcape software.
        \end{itemize}
\end{itemize}

\section{Windows}

You want to install MiKTeX distribution and it would be a clever thing to select
an option to install the missing packages only when their are actually needed in
the document. This will make you MiKTeX distribution smaller, but if you won't
have an internet access when you need a new package, then you might have some
inconveniences. Hence, the alternative of installing all the packages at once
might be a safer option.

\end{document}

%vim: tw=80:spell:spelllang=en_gb
