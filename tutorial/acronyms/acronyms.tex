\documentclass[a4paper,11pt]{article}
\usepackage[utf8x]{inputenc}
\usepackage[T1]{fontenc}
\usepackage{fullpage}
\linespread{1.1}
\usepackage{mathpazo}
\usepackage{verbatim}
\usepackage{listings}
\lstset{
%language=TeX,
breaklines=True}
\usepackage[version=3]{mhchem}
\usepackage[]{hyperref}

\usepackage[printonlyused]{acronym}
\usepackage[xindy]{glossaries}

\title{Using packages to use dynamic acronyms instead of hard-coding them}
\author{Ignas Anikevicius}

\begin{document}
\maketitle

% FIXME clumsy structure!!!!!!!!!!!!
In this document I will try to show you how to create and use a list of acronyms
in your document the \TeX\ way. One might ask: 'Why you need this? I can type
everything manually and it will be allright\ldots' I guess that if you right
your thesis which is 200 pages long, checking whether you have included the
definition of the acronym the first time you used a phrase is the least you
want.

\section{The acronym package}

\subsection{An example}

List of used acronyms:
\indent
\begin{acronym}
    \acro{d2o}[\ce{D2O}]{deuterated water}
    \acro{h2o}[\ce{H2O}]{water}
    \acro{dmf}[\ce{DMF}]{di-methyl-formamide}
    \acro{dmso}[\ce{DMSO}]{di-methyl-sulfoxide}
\end{acronym}

There are only 2 differences between \ac{d2o} and \ac{h2o}. On the other hand,
both \ac{d2o} and \ac{h2o} should behave chemically identically. Also
\ac{dmf} is in the acronym list.


\subsection{Code used for the example}

\lstinputlisting{acro.tex}

As you see, I have defined more acronyms than I have actually used. But, since I
have initiated the package with \verb|printonlyused| option (ie. I have
\verb|\usepackage[printonlyused]{acronym}| in my preamble.) However, it is
possible to get individual acronyms printed once \verb|\acused{}| command is
issued. Then the acronym will be printed regardless whether it is mentioned or
not as the package will think, that it was used. Also, only short version will
occur in the text if it will be used.

If the \verb|printonlyused| option is omitted, then all defined acronyms will be
printed out as a result.

\section{The glossaries package}

Glossaries package is another possible way to get a similar kind of
functionality, but it is a little bit more complicated than \verb|acronym|
package. I believe that it is indispensable in writing glossaries or
dictionaries, but for simple acronym usage it might be too heavy.

However, for people who are interested in trying out alternatives, please refer
to these web-pages and the documentation of the package.

\end{document}

% Editor configuration:
% vim: tw=80:spell:spelllang=en_gb
