In the early 80s, I was planning to write the Great American Concurrency
Book.  
%
I was a TeX user, so I would need a set of macros.  
%
I thought that, with a little extra effort, I could make my macros usable by
others.  
%
Don Knuth had begun issuing early releases of the current version of TeX,
and I figured I could write what would become its standard macro package.  
%
That was the beginning of LaTeX.  
%
I was planning to write a user manual, but it never occurred to me that
anyone would actually pay money for it.  
%
In 1983, Peter Gordon, an Addison-Wesley editor, and his colleagues visited
me at SRI.  
%
Here is his account of what happened. 

\quote{
    \em
    Our primary mission was to gather information for Addison-Wesley "to
    publish a computer-based document processing system specifically
    designed for scientists and engineers, in both academic and professional
    environments." 
    %
    This system was to be part of a series of related products (software,
    manuals, books) and services (database, production).  
    %
    (La)TeX was a candidate to be at the core of that system.  (I am quoting
    from the original business plan.)  
    %
    Fortunately, I did not listen to your doubt that anyone would buy the
    LaTeX manual, because more than a few hundred thousand people actually
    did.  
    %
    The exact number, of course, cannot accurately be determined, inasmuch
    as many people (not all friends and relatives) bought the book more than
    once, so heavily was it used.  
    }

Meanwhile, I still haven't written the Great American Concurrency Book.  
