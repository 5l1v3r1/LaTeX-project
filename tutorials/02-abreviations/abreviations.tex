% Clever thing to minimize the files I need to edit to change how the files are
% looking. The following one line comment is to trick VIM-LaTeX
\documentclass[
    draft
]{scrartcl}
%documentclass{scrartcl}
\KOMAoptions{
    fontsize=10pt
}
\setkomafont{pagenumber}{\bfseries\upshape\oldstylenums}
\renewcommand{\titlefont}{\rm\bfseries\LARGE}
\usepackage{
    ifxetex, 
    ifdraft,
    ifthen
}

\ifxetex
  \usepackage{fontspec}
  \usepackage{xunicode}
  \defaultfontfeatures{Mapping=tex-text} % To support LaTeX quoting style
  \setromanfont{Gentium}
\else
  \usepackage[utf8]{inputenc}
  \usepackage[T1]{fontenc}
  \usepackage{lmodern,textcomp}
\fi

\usepackage{
%    standalone,
%    lastpage,
    geometry,
    scrpage2,
%    setspace,
    amsmath,
    caption,
    calc,
%    floatrow,
}
\usepackage{
%    xcolor,
    graphicx,
    tikz,
    chemfig
}
\usepackage[final]{listings}
\usepackage[version=3]{mhchem}
\usepackage[update,verbose=false]{epstopdf}
\usepackage[colorinlistoftodos,obeyDraft]{todonotes}
\usepackage{hyperref}


% Set the geometry
\geometry{
    paper = a4paper,
    top=3cm,
    bottom=4cm,
    footskip=1cm,
    marginparwidth=3.5cm,
    headsep=1cm
}
\ifoptiondraft{
\geometry{inner=1.5cm, outer=4cm}
}{
\geometry{inner=3.0cm, outer=2.5cm}
}

%\onehalfspacing

% Setup hyperref
\hypersetup{
    colorlinks,
    urlcolor=blue,
    breaklinks
}

\usetikzlibrary{
    arrows,
    decorations.pathmorphing,
    backgrounds,
    positioning,
    fit,
    petri
}

% Define collors
\definecolor{myyellow}{HTML}{FFFAC9}
\definecolor{myyellowl}{HTML}{FFFBDD}

% Define lstlisting env
\lstset{
    language=[LaTeX]TeX,
    backgroundcolor=\color{myyellowl},
    numbers=left,
    numberstyle=\footnotesize,
    breaklines=true,
    breakatwhitespace=true,
    print=true
}

% Renew 2 styles
\renewpagestyle{plain}{{}{}{}}{{}{}
{\hfill\pagemark{}}}
\renewpagestyle{headings}{{}{}{}}{{}{}
{\hfill\pagemark{}}}

% Set the headings page style
\pagestyle{headings}

% Text mode commands
\newcommand{\uurl}[2]{\href{#1}{#2}\footnote{The URL is \url{#1}}}
\newcommand{\ftype}[1]{\texttt{.#1}}
\newcommand{\fname}[2]{\texttt{#1.#2}}
\newcommand{\pkg}[1]{\texttt{#1}}
\newcommand{\env}[1]{\texttt{#1}}
\newcommand{\cmd}[1]{\texttt{\textbackslash{}#1}}
\newcommand{\usepkg}[2]{
    \texttt{\textbackslash{}usepackage%
    \ifthenelse{\equal{#2}{}}{}{[#2]}\{#1\}}}
\newcommand{\comment}[1]{}

% New commands which ease the work. Units and relative uncertainties
\newcommand{\unit}[1]{\ensuremath{\, \mathrm{#1}}}
\newcommand{\rel}[1]{\ensuremath{ \cfrac{\Delta #1}{#1}}}
\newcommand{\eten}[1]{\ensuremath{ \times 10^{#1}}}
\newcommand{\DP}[2]{\ensuremath{\cfrac{\partial #1}{\partial #2}}}
\newcommand{\DD}[2]{\ensuremath{\cfrac{\mathrm{d} #1}{\mathrm{d} #2}}}
\newcommand{\dd}[1]{\ensuremath{\mathrm{d}#1}}

% Alter some LaTeX defaults for better treatment of figures:
    % See p.105 of "TeX Unbound" for suggested values.
    % See pp. 199-200 of Lamport's "LaTeX" book for details.
    %   General parameters, for ALL pages:
    \renewcommand{\topfraction}{0.9}    % max fraction of floats at top
    \renewcommand{\bottomfraction}{0.8} % max fraction of floats at bottom
    %   Parameters for TEXT pages (not float pages):
    \setcounter{topnumber}{2}
    \setcounter{bottomnumber}{2}
    \setcounter{totalnumber}{4}     % 2 may work better
    \setcounter{dbltopnumber}{2}    % for 2-column pages
    \renewcommand{\dbltopfraction}{0.9} % fit big float above 2-col. text
    \renewcommand{\textfraction}{0.07}  % allow minimal text w. figs
    %   Parameters for FLOAT pages (not text pages):
    \renewcommand{\floatpagefraction}{0.7}      % require fuller float pages
    % N.B.: floatpagefraction MUST be less than topfraction !!
    \renewcommand{\dblfloatpagefraction}{0.7}   % require fuller float pages

\captionsetup{
    format          = plain,        %
    labelformat     = simple,       %
    labelsep        = period,       %
    justification   = default,      %
    font            = default,      %
    labelfont       = {bf,sf},      %
    textfont        = default,      %
    margin          = 0pt,          %
    indention       = 0pt,          %
    parindent       = 0pt,          %
    hangindent      = 0pt,          %
    singlelinecheck = false         %
}

\renewcommand{\thefigure}{\oldstylenums{\arabic{figure}}}

\setatomsep{5mm}
\setbondoffset{.5mm}
\setcrambond{2.5pt}{1pt}{2pt}
\setbondstyle{thick}
\renewcommand*\printatom[1]{{\footnotesize\ensuremath{\mathsf{#1}}}}


% Custom things
\usepackage[version=3]{mhchem}
\usepackage[printonlyused]{acronym}
\usepackage[xindy]{glossaries}

\title{Using packages to use dynamic acronyms instead of hard-coding them}
\author{Ignas Anikevicius}

\begin{document}
\maketitle
\tableofcontents
\listoftodos

%
In this document I will try to show you how to create and use a list of acronyms
    in your document the \TeX{} way.
%
One might ask: 'Why you need this? I can type everything manually and it will be
    allright\ldots'
%
I guess that if you write your thesis which is 200 pages long, checking whether
    you have included the definition of the acronym the first time you used a
    phrase is the least you want.

% ----------------------------------------------------------------------
\section{The \pkg{acronym} package}
% ----------------------------------------------------------------------

%
\subsection{An example}

%
List of used acronyms:
\indent
\begin{acronym}
    \acro{d2o}[\ce{D2O}]{deuterated water}
    \acro{h2o}[\ce{H2O}]{water}
    \acro{dmf}[\ce{DMF}]{di-methyl-formamide}
    \acro{dmso}[\ce{DMSO}]{di-methyl-sulfoxide}
\end{acronym}

There are only 2 differences between \ac{d2o} and \ac{h2o}. On the other hand,
both \ac{d2o} and \ac{h2o} should behave chemically identically. Also
\ac{dmf} is in the acronym list.


%
\subsection{Code used for the example}

%
\lstinputlisting{acro.tex}

%
As you see, I have defined more acronyms than I have actually used.
%
But, since I have initiated the package with \verb|printonlyused| option as
    follows:
    %
\begin{lstlisting}
\usepackage[printonlyused]{acronym}
\end{lstlisting}
    %
    I do not have to worry about unused acronyms being printed in the list.

%
If by any chance you want some acronyms to be printed out without them using in
    the text, you can achieve it by issuing \verb|\acused{}| command, where in
    the brackets you put the respective label of the acronym.
%
Then the acronym will be printed regardless whether it is mentioned or
    not as the package will think, that it was used.
%
Also, only short version will occur in the text if it will be used after issuing
    the mentioned command.

%
If the \verb|printonlyused| option is omitted, then all defined acronyms will be
    printed out as a result.

% ----------------------------------------------------------------------
\section{The \pkg{glossaries} package}
% ----------------------------------------------------------------------

%
Glossaries package is another possible way to get a similar kind of
    functionality.
%
It has much more functionality and is probably more suited towards people who
    want to create extensive lists and have more than one glossary at a time.
%
With this extensibility comes complexity as it is slightly harder to get the
    glossaries being generated as you need to open command line shell and
    execute some commands.
%
However for simple acronym usage the \pkg{acronym} package is more than enough.
%
But for people who are interested in trying out alternatives, please refer
    to these web-pages and the documentation of the package.

\end{document}

% Editor configuration:
% vim: tw=80:spell:spelllang=en_gb
