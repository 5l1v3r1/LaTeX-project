% Clever thing to minimize the files I need to edit to change how the files are
% looking. The following one line comment is to trick VIM-LaTeX
% \documentclass{article}
%documentclass{scrartcl}
\KOMAoptions{
    fontsize=10pt
}
\setkomafont{pagenumber}{\bfseries\upshape\oldstylenums}
\renewcommand{\titlefont}{\rm\bfseries\LARGE}
\usepackage{
    ifxetex, 
    ifdraft,
    ifthen
}

\ifxetex
  \usepackage{fontspec}
  \usepackage{xunicode}
  \defaultfontfeatures{Mapping=tex-text} % To support LaTeX quoting style
  \setromanfont{Gentium}
\else
  \usepackage[utf8]{inputenc}
  \usepackage[T1]{fontenc}
  \usepackage{lmodern,textcomp}
\fi

\usepackage{
%    standalone,
%    lastpage,
    geometry,
    scrpage2,
%    setspace,
    amsmath,
    caption,
    calc,
%    floatrow,
}
\usepackage{
%    xcolor,
    graphicx,
    tikz,
    chemfig
}
\usepackage[final]{listings}
\usepackage[version=3]{mhchem}
\usepackage[update,verbose=false]{epstopdf}
\usepackage[colorinlistoftodos,obeyDraft]{todonotes}
\usepackage{hyperref}


% Set the geometry
\geometry{
    paper = a4paper,
    top=3cm,
    bottom=4cm,
    footskip=1cm,
    marginparwidth=3.5cm,
    headsep=1cm
}
\ifoptiondraft{
\geometry{inner=1.5cm, outer=4cm}
}{
\geometry{inner=3.0cm, outer=2.5cm}
}

%\onehalfspacing

% Setup hyperref
\hypersetup{
    colorlinks,
    urlcolor=blue,
    breaklinks
}

\usetikzlibrary{
    arrows,
    decorations.pathmorphing,
    backgrounds,
    positioning,
    fit,
    petri
}

% Define collors
\definecolor{myyellow}{HTML}{FFFAC9}
\definecolor{myyellowl}{HTML}{FFFBDD}

% Define lstlisting env
\lstset{
    language=[LaTeX]TeX,
    backgroundcolor=\color{myyellowl},
    numbers=left,
    numberstyle=\footnotesize,
    breaklines=true,
    breakatwhitespace=true,
    print=true
}

% Renew 2 styles
\renewpagestyle{plain}{{}{}{}}{{}{}
{\hfill\pagemark{}}}
\renewpagestyle{headings}{{}{}{}}{{}{}
{\hfill\pagemark{}}}

% Set the headings page style
\pagestyle{headings}

% Text mode commands
\newcommand{\uurl}[2]{\href{#1}{#2}\footnote{The URL is \url{#1}}}
\newcommand{\ftype}[1]{\texttt{.#1}}
\newcommand{\fname}[2]{\texttt{#1.#2}}
\newcommand{\pkg}[1]{\texttt{#1}}
\newcommand{\env}[1]{\texttt{#1}}
\newcommand{\cmd}[1]{\texttt{\textbackslash{}#1}}
\newcommand{\usepkg}[2]{
    \texttt{\textbackslash{}usepackage%
    \ifthenelse{\equal{#2}{}}{}{[#2]}\{#1\}}}
\newcommand{\comment}[1]{}

% New commands which ease the work. Units and relative uncertainties
\newcommand{\unit}[1]{\ensuremath{\, \mathrm{#1}}}
\newcommand{\rel}[1]{\ensuremath{ \cfrac{\Delta #1}{#1}}}
\newcommand{\eten}[1]{\ensuremath{ \times 10^{#1}}}
\newcommand{\DP}[2]{\ensuremath{\cfrac{\partial #1}{\partial #2}}}
\newcommand{\DD}[2]{\ensuremath{\cfrac{\mathrm{d} #1}{\mathrm{d} #2}}}
\newcommand{\dd}[1]{\ensuremath{\mathrm{d}#1}}

% Alter some LaTeX defaults for better treatment of figures:
    % See p.105 of "TeX Unbound" for suggested values.
    % See pp. 199-200 of Lamport's "LaTeX" book for details.
    %   General parameters, for ALL pages:
    \renewcommand{\topfraction}{0.9}    % max fraction of floats at top
    \renewcommand{\bottomfraction}{0.8} % max fraction of floats at bottom
    %   Parameters for TEXT pages (not float pages):
    \setcounter{topnumber}{2}
    \setcounter{bottomnumber}{2}
    \setcounter{totalnumber}{4}     % 2 may work better
    \setcounter{dbltopnumber}{2}    % for 2-column pages
    \renewcommand{\dbltopfraction}{0.9} % fit big float above 2-col. text
    \renewcommand{\textfraction}{0.07}  % allow minimal text w. figs
    %   Parameters for FLOAT pages (not text pages):
    \renewcommand{\floatpagefraction}{0.7}      % require fuller float pages
    % N.B.: floatpagefraction MUST be less than topfraction !!
    \renewcommand{\dblfloatpagefraction}{0.7}   % require fuller float pages

\captionsetup{
    format          = plain,        %
    labelformat     = simple,       %
    labelsep        = period,       %
    justification   = default,      %
    font            = default,      %
    labelfont       = {bf,sf},      %
    textfont        = default,      %
    margin          = 0pt,          %
    indention       = 0pt,          %
    parindent       = 0pt,          %
    hangindent      = 0pt,          %
    singlelinecheck = false         %
}

\renewcommand{\thefigure}{\oldstylenums{\arabic{figure}}}

\setatomsep{5mm}
\setbondoffset{.5mm}
\setcrambond{2.5pt}{1pt}{2pt}
\setbondstyle{thick}
\renewcommand*\printatom[1]{{\footnotesize\ensuremath{\mathsf{#1}}}}


% Custom things
\usepackage[version=3]{mhchem}
\usepackage[]{amsmath}

\title{Getting nice equations in the text}
\author{Ignas Anikevicius}

\begin{document}

\maketitle

In this short tutorial I will explain how to make simple equations. This will
include chemical equations as well. In order to get the required packages in
order to do these things, please look in to the source (.tex) file of this
document. If you feel, that you need more information on typesetting various
mathematical formulas, please refer to this online 
\href{https://secure.wikimedia.org/wikibooks/en/wiki/LaTeX/Mathematics}{book}.

\section{Single-line mathematical equations}

Many might want to have numbered equations in the text. For example you are
writing a lot of text and then you want to make \textbf{one} numbered equation,
which would start at a new line. As follows:
\begin{equation}
    \dd f = \DP f x \dd x + \DP f y \dd y
    \label{eq:differential}
\end{equation}


Then you can refer to the equation \ref{eq:differential} by using a simple
\verb|\ref{eq:differential}| command.

The code required to get equation \ref{eq:differential} is as follows (NB I used
custom commands to simplify the code. On making such commands, please read
another document):
\lstinputlisting{eg-smath1.tex}

Just for completeness, this code would need to be written down without my simple
custom additions:
\lstinputlisting{eg-smath11.tex}

No we now how to produce a numbered equation. What about unnumbered? It is very
easy, we can just \verb|*| symbols as follows:
\lstinputlisting{eg-smath2.tex}
And the result is (NB we do not need label any more, since this is an unnumbered
equations and we \textbf{do not} intend to refer it later on in the text):
\begin{equation*}
    \dd f = \DP f x \dd x + \DP f y \dd y
\end{equation*}


There is another way to get this effect by using the following code (\verb|\[|
and \verb|\]| can be replaced with \verb|$$|, however the first variant shows
more clarity as to where the environment starts and finishes):
\lstinputlisting{eg-smath3.tex}

And the result is as follows:
\[
    \dd f = \DP f x \dd x + \DP f y \dd y
\]


\section{Multiple-line mathematical equations}

Many line equations are slightly different to produce. Many probably know the
environments called \verb|eqnarray| and \verb|eqnarray*|, which can be used to
produce the following set of equations (the starred version will produce
unnumbered version as before):
\begin{eqnarray}
    \dd f &=& \DP f x \dd x + \DP f y \dd y
    \label{eq:exact_multi1}
    \\
    \dd g &=& \DP g x \dd x + \DP g y \dd y + \DP g z \dd z
    \label{eq:exact_multi2}
    \\
    \dd h = \DP h x \dd x &+& \DP h y \dd y
    \label{eq:exact_multi3}
\end{eqnarray}

with code:
\lstinputlisting{eg-mmath1.tex}

However, one can notice, that the spacing around \verb|=| sign is not perfect
(although some might like it). A much better approach is to use \verb|align|
environment, which has even more capabilities.
\begin{align}
    \dd f &= \DP f x \dd x + \DP f y \dd y
    \nonumber
    \\
    \dd g &= \DP g x \dd x + \DP g y \dd y + \DP g z \dd z
    \label{eq:exact_align}
    \\
    \dd h = \DP h x \dd x &+ \DP h y \dd y
    \nonumber
\end{align}

The code to generate the equations above is (NB different syntax!)
\lstinputlisting{eg-mmath2.tex}

\section{Chemical equations}

Now we come to a more interesting topic. How to typeset chemical equations
quickly? Suppose we have an equation as follows:
\[
\mathrm{6 KSCN + FeCl_3 \rightarrow K_3[Fe(SCN)_6] + 3 KCl}
\]


One way to typeset it would be to use existing commands and the code required
would be as follows:
\lstinputlisting{eg-chem1.tex}

It is not bad, but if we have a following equation:
\[
\mathrm{Ba^{2+}(aq) + SO_4^{2-}(aq) \rightarrow BaSO_4\downarrow}
\]

would require you to type the following:
\lstinputlisting{eg-chem2.tex}

And this might complicate the things even more:
\[
\mathrm{Ba^{2+}(aq) + SO_4^{2-}(aq)
\underset{bellow}{\overset{above}{\longrightarrow}} BaSO_4\downarrow}
\]

\lstinputlisting{eg-chem3.tex}

As you see, the more complicated the equation gets, the harder it is to fulfil
all the requirements. For example you can see, that the last equation does not
have a properly spaced arrow, which would be very nice to have.

However, it is very nice, that \LaTeX\ is very extensible and open source and
a lot of people have contributed their time and knowledge to get it work as
\textbf{we} want and need. For chemical reactions and other symbols typesetting
I suggest you using the \verb|mhchem| package which should already be bundled with your
\TeX\ distribution. To start using it just include
\verb|\usepackage[version=3]{mhchem}| into your preamble.

The following code using this package produce better formatted equations.
\lstinputlisting{eg-mhchem.tex}
and we get:
\begin{align}
    \ce{6KSCN + FeCl3 -> K3[Fe(SCN)6] + 3KCl}
    \nonumber
    \\
    \ce{Ag+(aq) + Cl-(aq) -> AgCl v}
    \nonumber
    \\
    \ce{Ba^{2+}(aq) + SO4^{2-}(aq) -> BaSO4 v}
    \\
    \ce{Ba^{2+}(aq) + SO4^{2-}(aq) ->[\mathrm{above}][\mathrm{bellow}] BaSO4 v}
    \nonumber
\end{align}


For more capabilities of the package it would be possibly best to flick through
through the documentation of the package which can be found following this link.

\end{document}

% Editor configuration:
% vim: tw=80:spell:spelllang=en_gb
