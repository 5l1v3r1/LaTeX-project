% Clever thing to minimize the files I need to edit to change how the files are
% looking. The following one line comment is to trick VIM-LaTeX
\documentclass[
%    draft
]{scrartcl}
%documentclass{scrartcl}
\KOMAoptions{
    fontsize=10pt
}
\setkomafont{pagenumber}{\bfseries\upshape\oldstylenums}
\renewcommand{\titlefont}{\rm\bfseries\LARGE}
\usepackage{
    ifxetex, 
    ifdraft,
    ifthen
}

\ifxetex
  \usepackage{fontspec}
  \usepackage{xunicode}
  \defaultfontfeatures{Mapping=tex-text} % To support LaTeX quoting style
  \setromanfont{Gentium}
\else
  \usepackage[utf8]{inputenc}
  \usepackage[T1]{fontenc}
  \usepackage{lmodern,textcomp}
\fi

\usepackage{
%    standalone,
%    lastpage,
    geometry,
    scrpage2,
%    setspace,
    amsmath,
    caption,
    calc,
%    floatrow,
}
\usepackage{
%    xcolor,
    graphicx,
    tikz,
    chemfig
}
\usepackage[final]{listings}
\usepackage[version=3]{mhchem}
\usepackage[update,verbose=false]{epstopdf}
\usepackage[colorinlistoftodos,obeyDraft]{todonotes}
\usepackage{hyperref}


% Set the geometry
\geometry{
    paper = a4paper,
    top=3cm,
    bottom=4cm,
    footskip=1cm,
    marginparwidth=3.5cm,
    headsep=1cm
}
\ifoptiondraft{
\geometry{inner=1.5cm, outer=4cm}
}{
\geometry{inner=3.0cm, outer=2.5cm}
}

%\onehalfspacing

% Setup hyperref
\hypersetup{
    colorlinks,
    urlcolor=blue,
    breaklinks
}

\usetikzlibrary{
    arrows,
    decorations.pathmorphing,
    backgrounds,
    positioning,
    fit,
    petri
}

% Define collors
\definecolor{myyellow}{HTML}{FFFAC9}
\definecolor{myyellowl}{HTML}{FFFBDD}

% Define lstlisting env
\lstset{
    language=[LaTeX]TeX,
    backgroundcolor=\color{myyellowl},
    numbers=left,
    numberstyle=\footnotesize,
    breaklines=true,
    breakatwhitespace=true,
    print=true
}

% Renew 2 styles
\renewpagestyle{plain}{{}{}{}}{{}{}
{\hfill\pagemark{}}}
\renewpagestyle{headings}{{}{}{}}{{}{}
{\hfill\pagemark{}}}

% Set the headings page style
\pagestyle{headings}

% Text mode commands
\newcommand{\uurl}[2]{\href{#1}{#2}\footnote{The URL is \url{#1}}}
\newcommand{\ftype}[1]{\texttt{.#1}}
\newcommand{\fname}[2]{\texttt{#1.#2}}
\newcommand{\pkg}[1]{\texttt{#1}}
\newcommand{\env}[1]{\texttt{#1}}
\newcommand{\cmd}[1]{\texttt{\textbackslash{}#1}}
\newcommand{\usepkg}[2]{
    \texttt{\textbackslash{}usepackage%
    \ifthenelse{\equal{#2}{}}{}{[#2]}\{#1\}}}
\newcommand{\comment}[1]{}

% New commands which ease the work. Units and relative uncertainties
\newcommand{\unit}[1]{\ensuremath{\, \mathrm{#1}}}
\newcommand{\rel}[1]{\ensuremath{ \cfrac{\Delta #1}{#1}}}
\newcommand{\eten}[1]{\ensuremath{ \times 10^{#1}}}
\newcommand{\DP}[2]{\ensuremath{\cfrac{\partial #1}{\partial #2}}}
\newcommand{\DD}[2]{\ensuremath{\cfrac{\mathrm{d} #1}{\mathrm{d} #2}}}
\newcommand{\dd}[1]{\ensuremath{\mathrm{d}#1}}

% Alter some LaTeX defaults for better treatment of figures:
    % See p.105 of "TeX Unbound" for suggested values.
    % See pp. 199-200 of Lamport's "LaTeX" book for details.
    %   General parameters, for ALL pages:
    \renewcommand{\topfraction}{0.9}    % max fraction of floats at top
    \renewcommand{\bottomfraction}{0.8} % max fraction of floats at bottom
    %   Parameters for TEXT pages (not float pages):
    \setcounter{topnumber}{2}
    \setcounter{bottomnumber}{2}
    \setcounter{totalnumber}{4}     % 2 may work better
    \setcounter{dbltopnumber}{2}    % for 2-column pages
    \renewcommand{\dbltopfraction}{0.9} % fit big float above 2-col. text
    \renewcommand{\textfraction}{0.07}  % allow minimal text w. figs
    %   Parameters for FLOAT pages (not text pages):
    \renewcommand{\floatpagefraction}{0.7}      % require fuller float pages
    % N.B.: floatpagefraction MUST be less than topfraction !!
    \renewcommand{\dblfloatpagefraction}{0.7}   % require fuller float pages

\captionsetup{
    format          = plain,        %
    labelformat     = simple,       %
    labelsep        = period,       %
    justification   = default,      %
    font            = default,      %
    labelfont       = {bf,sf},      %
    textfont        = default,      %
    margin          = 0pt,          %
    indention       = 0pt,          %
    parindent       = 0pt,          %
    hangindent      = 0pt,          %
    singlelinecheck = false         %
}

\renewcommand{\thefigure}{\oldstylenums{\arabic{figure}}}

\setatomsep{5mm}
\setbondoffset{.5mm}
\setcrambond{2.5pt}{1pt}{2pt}
\setbondstyle{thick}
\renewcommand*\printatom[1]{{\footnotesize\ensuremath{\mathsf{#1}}}}


% Custom packages

\title{Posters and \LaTeX{}}
\author{Ignas Anikevičius}

\begin{document}

\maketitle
\tableofcontents
\listoftodos
\vskip 1em

%
This tutorial quickly goes through several packages which can be used to make
    posters.
%
Since there are a lot of ways of doing posters in \LaTeX{}, I feel that
    information resources on this topic are somehow scattered and there is no
    one good way of making posters as almost all of them have at least several
    drawbacks.
%
I will try to inform you about these drawbacks and benefits here, but you should
    definitely search for information yourself and decide on your own.

% ----------------------------------------------------------------------
\section{KOMA script article class - \pkg{scrartcl}}
% ----------------------------------------------------------------------

%
This way would involve heavy tweaking of the standard KOMA script article class
    - \pkg{scrartcl}.
%
The reason behind the choice of this document class is that it provides a lot of
    customizability for things such as font selection and others.
%
This mainly would involve scaling all the fonts to the right size and the fonts
    which should be used for this job would be Times/Helvetica/Courier
    combination, which will always be a conservative, but wise choice.

%
The concept itself might sound as not very demanding and time consuming, but I
    can assure you, that trying to produce a well looking document of A0 paper
    size might not be very comfortable as the article classes are suited for
    paper sizes comparable with A4.
%
A solution would be to produce the poster as a A4 page size document and then
    scale it up to the required size.
%
However, might cause different problems as $1\unit{cm}$ on the paper will not be
    equal to $1\unit{cm}$ on the poster and this might make the poster production
    very cumbersome.

%
Since there are a lot of issues involved with making posters with standard
    document classes, one should reconsider the idea before trying it out.
    \todo{Finish it!}

% ----------------------------------------------------------------------
\section{The \pkg{a0poster} document class}
% ----------------------------------------------------------------------

%
This is probably the main package for \LaTeX{} posters and there are many other
    posters, which have been inspired or are based on this package.
%
There is a really extensive tutorial if you follow this 
    \uurl{http://theoval.cmp.uea.ac.uk/~nlct/latex/posters/index.html}{link} and
    there is even another website, which uses this document class and
    \pkg{TikZ} package to create quite impressive graphics.

% ----------------------------------------------------------------------
\section{The \pkg{baposter} package}
% ----------------------------------------------------------------------

%
This package seems to be a good alternative to the \pkg{a0poster} and the
    project's homepage contains a lot of example posters, which were presumably
    made by the author of the package.
%
The results seem to be good and this might be really worth looking at.
%
However, I should warn you, that this package doesn't seem to be accepted to the
    CTAN website, which might not make people happy, who are very much concerned
    with the out-of-the-box experience once they install their \TeX{}
    distribution.

%
The website can be found by following
    \uurl{http://www.brian-amberg.de/uni/poster/}{this link}.

% ----------------------------------------------------------------------
\section{The \pkg{beamer} document class and the \pkg{beamerposter} package}
% ----------------------------------------------------------------------

%
The \pkg{beamer} document class is very useful in making presentation slides,
    but its flexibility allows it to be a good solution for poster production
%
Since \pkg{beamer} uses the \pkg{TikZ} package a lot, it can draw those fancy
    frames for blocks of text in the poster.
%
However, the \pkg{beamer} class itself still might lack some functionality which
    would make poster making much more convenient, but luckily Philippe Dreuw
    and Thomas Deselaers have created the \pkg{beamerposter} package, which is
    just an extension of \pkg{beamer} and \pkg{a0poster} classes.

%
One can find examples and more information on their
    \uurl{http://www-i6.informatik.rwth-aachen.de/~dreuw/latexbeamerposter.php}{website}.
%
For most people there should be no need to install the \pkg{beamerposter}
    package separately as it is included in the latest \TeX{} distributions
    (\TeX{} Live, Mac\TeX{} and MiK\TeX{}).
%
However, the CTAN folder for this package can be found
    \uurl{http://www.ctan.org/tex-archive/macros/latex/contrib/beamerposter}{here}.

%
Other useful links for this package:
\begin{itemize}
    \item \uurl{http://robjhyndman.com/researchtips/beamer-poster/}{Rob J
        Hyndman blog ``Research Tips''};
    \item \uurl{http://robjhyndman.com/researchtips/beamer-poster/}{Beamerposter
        Google Group};
\end{itemize}

% ----------------------------------------------------------------------
\section{Other packages}
% ----------------------------------------------------------------------

%
Of course there are several other packages and it probably would be a good idea
    just to be aware of them so that when you \emph{need} an alternative, you
    have a choice.
%
If you find the list incomplete, please let me know, and I will update the
    document accordingly or please submit a patch to the git repository.

\begin{itemize}
    \item 
        The \pkg{sciposter} package
            \uurl{http://www.ctan.org/tex-archive/macros/latex/contrib/sciposter/}{CTAN
            directory} shows, that the last update to this package was in 2006.
        %
        Therefore, I guess, that it would be wiser to go with the alternative
            \pkg{beamerposter} package.
        %
        On the other hand, for people who are searching for some solutions, the
            source code of the package might prove to be very valuable and it
            might be able to adapt it to get even more functionality.
        %
\end{itemize}


\end{document}

% Editor configuration:
% vim: tw=80:spell:spelllang=en_gb
