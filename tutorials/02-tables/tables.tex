% Clever thing to minimize the files I need to edit to change how the files are
% looking. The following one line comment is to trick VIM-LaTeX
% \documentclass{article}
%documentclass{scrartcl}
\KOMAoptions{
    fontsize=10pt
}
\setkomafont{pagenumber}{\bfseries\upshape\oldstylenums}
\renewcommand{\titlefont}{\rm\bfseries\LARGE}
\usepackage{
    ifxetex, 
    ifdraft,
    ifthen
}

\ifxetex
  \usepackage{fontspec}
  \usepackage{xunicode}
  \defaultfontfeatures{Mapping=tex-text} % To support LaTeX quoting style
  \setromanfont{Gentium}
\else
  \usepackage[utf8]{inputenc}
  \usepackage[T1]{fontenc}
  \usepackage{lmodern,textcomp}
\fi

\usepackage{
%    standalone,
%    lastpage,
    geometry,
    scrpage2,
%    setspace,
    amsmath,
    caption,
    calc,
%    floatrow,
}
\usepackage{
%    xcolor,
    graphicx,
    tikz,
    chemfig
}
\usepackage[final]{listings}
\usepackage[version=3]{mhchem}
\usepackage[update,verbose=false]{epstopdf}
\usepackage[colorinlistoftodos,obeyDraft]{todonotes}
\usepackage{hyperref}


% Set the geometry
\geometry{
    paper = a4paper,
    top=3cm,
    bottom=4cm,
    footskip=1cm,
    marginparwidth=3.5cm,
    headsep=1cm
}
\ifoptiondraft{
\geometry{inner=1.5cm, outer=4cm}
}{
\geometry{inner=3.0cm, outer=2.5cm}
}

%\onehalfspacing

% Setup hyperref
\hypersetup{
    colorlinks,
    urlcolor=blue,
    breaklinks
}

\usetikzlibrary{
    arrows,
    decorations.pathmorphing,
    backgrounds,
    positioning,
    fit,
    petri
}

% Define collors
\definecolor{myyellow}{HTML}{FFFAC9}
\definecolor{myyellowl}{HTML}{FFFBDD}

% Define lstlisting env
\lstset{
    language=[LaTeX]TeX,
    backgroundcolor=\color{myyellowl},
    numbers=left,
    numberstyle=\footnotesize,
    breaklines=true,
    breakatwhitespace=true,
    print=true
}

% Renew 2 styles
\renewpagestyle{plain}{{}{}{}}{{}{}
{\hfill\pagemark{}}}
\renewpagestyle{headings}{{}{}{}}{{}{}
{\hfill\pagemark{}}}

% Set the headings page style
\pagestyle{headings}

% Text mode commands
\newcommand{\uurl}[2]{\href{#1}{#2}\footnote{The URL is \url{#1}}}
\newcommand{\ftype}[1]{\texttt{.#1}}
\newcommand{\fname}[2]{\texttt{#1.#2}}
\newcommand{\pkg}[1]{\texttt{#1}}
\newcommand{\env}[1]{\texttt{#1}}
\newcommand{\cmd}[1]{\texttt{\textbackslash{}#1}}
\newcommand{\usepkg}[2]{
    \texttt{\textbackslash{}usepackage%
    \ifthenelse{\equal{#2}{}}{}{[#2]}\{#1\}}}
\newcommand{\comment}[1]{}

% New commands which ease the work. Units and relative uncertainties
\newcommand{\unit}[1]{\ensuremath{\, \mathrm{#1}}}
\newcommand{\rel}[1]{\ensuremath{ \cfrac{\Delta #1}{#1}}}
\newcommand{\eten}[1]{\ensuremath{ \times 10^{#1}}}
\newcommand{\DP}[2]{\ensuremath{\cfrac{\partial #1}{\partial #2}}}
\newcommand{\DD}[2]{\ensuremath{\cfrac{\mathrm{d} #1}{\mathrm{d} #2}}}
\newcommand{\dd}[1]{\ensuremath{\mathrm{d}#1}}

% Alter some LaTeX defaults for better treatment of figures:
    % See p.105 of "TeX Unbound" for suggested values.
    % See pp. 199-200 of Lamport's "LaTeX" book for details.
    %   General parameters, for ALL pages:
    \renewcommand{\topfraction}{0.9}    % max fraction of floats at top
    \renewcommand{\bottomfraction}{0.8} % max fraction of floats at bottom
    %   Parameters for TEXT pages (not float pages):
    \setcounter{topnumber}{2}
    \setcounter{bottomnumber}{2}
    \setcounter{totalnumber}{4}     % 2 may work better
    \setcounter{dbltopnumber}{2}    % for 2-column pages
    \renewcommand{\dbltopfraction}{0.9} % fit big float above 2-col. text
    \renewcommand{\textfraction}{0.07}  % allow minimal text w. figs
    %   Parameters for FLOAT pages (not text pages):
    \renewcommand{\floatpagefraction}{0.7}      % require fuller float pages
    % N.B.: floatpagefraction MUST be less than topfraction !!
    \renewcommand{\dblfloatpagefraction}{0.7}   % require fuller float pages

\captionsetup{
    format          = plain,        %
    labelformat     = simple,       %
    labelsep        = period,       %
    justification   = default,      %
    font            = default,      %
    labelfont       = {bf,sf},      %
    textfont        = default,      %
    margin          = 0pt,          %
    indention       = 0pt,          %
    parindent       = 0pt,          %
    hangindent      = 0pt,          %
    singlelinecheck = false         %
}

\renewcommand{\thefigure}{\oldstylenums{\arabic{figure}}}

\setatomsep{5mm}
\setbondoffset{.5mm}
\setcrambond{2.5pt}{1pt}{2pt}
\setbondstyle{thick}
\renewcommand*\printatom[1]{{\footnotesize\ensuremath{\mathsf{#1}}}}


% Custome packages
%\usepackage{framed}
%\definecolor{shadecolor}{HTML}{F7ECE1}
%\definecolor{shadecolor2}{HTML}{F2DDC8}
\usepackage[]{amsmath}
\usepackage{blindtext}
\usepackage{multicol}
\usepackage[english,british]{babel}

\usepackage{stfloats}
\usepackage{booktabs}
\usepackage{array}

\title{Getting nice tables in the text}
\author{Ignas Anikevicius}

\begin{document}

\maketitle

Tables is usually the most difficult thing to do properly in \LaTeX . It is very
hard to cover everything needed to get nice tables in your document, so if you
feel, that you have not found your answers in this document, please refer to
this 
\uurl{https://secure.wikimedia.org/wikibooks/en/wiki/LaTeX/Tables\#The_table_environment_-_captioning_etc}{book}.

\tableofcontents

\section{Simple table usage}

Here I will give two examples of table making. One is using the internal \LaTeX\
table-making framework. Another is using a \verb|booktabs| package, which seems
to provide better looking tables.

\begin{table}[h]
    \centering
    \begin{tabular}{ccc}
        \hline
        First column & Second column & $\mathrm{3^{rd}}$ column
        \tabularnewline\hline
        1 & 2 & 3
        \tabularnewline
        1 & 2 & 3
        \tabularnewline\hline
    \end{tabular}
    \caption{A table with default separators.}
    \label{tab:table1}
\end{table}

\begin{table}[h]
    \centering
    \caption{The same table, just using booktabs package and the
        separators, which are provided by this package.}
    \begin{tabular}{ccc}
        \toprule
        First column & Second column & $\mathrm{3^{rd}}$ column
        \tabularnewline\midrule
        1 & 2 & 3
        \tabularnewline
        1 & 2 & 3
        \tabularnewline\bottomrule
    \end{tabular}
    \label{tab:table2}
\end{table}


As you can see, Table \ref{tab:table2} looks far better, because of different
spacing and line widths. The tables were produced using the code shown bellow:
\lstinputlisting{tab1.tex}
\lstinputlisting{tab2.tex}

\section{Multiple lines in table cells}

The basic method of constructing tables which was covered in the previous
section is good only if you do not need to specify the width of the columns and
you do not need to break the text inside a cell.

Suppose we add one more column into our table and make use of the ability in
provided by the \verb|array| package to get any environments working inside a
cell. To get more information on this please refer to this 
\uurl{https://secure.wikimedia.org/wikibooks/en/wiki/LaTeX/Tables\#The_table_environment_-_captioning_etc}{book}.

\begin{table}[h]
    \centering
    \caption{Table which shows the usage of getting various environments working inside a column.}
    \begin{tabular}{>{\centering}m{1.4cm} >{\centering}p{1.4cm} >{$}c<{$} c}
        \toprule
        First column & Second column & \mathrm{3^{rd}}\text{ column} & $\mathrm{4^{th}}$ column
        \tabularnewline\midrule
        1 & 2 & \delta & 4
        \tabularnewline
        1 & 2 & \epsilon & 4
        \tabularnewline\bottomrule
    \end{tabular}
    \label{tab:table4}
\end{table}

\lstinputlisting{tab4.tex}

\section{Tables in documents with more than one column}

\begin{multicols}{2}
    \begin{table*}[tpb]
    \centering
    \caption{A table in table* environment to span over the whole page.}
    \begin{tabular}{p{2cm} p{2cm} p{2cm} p{2cm}}
        \toprule
        First column & Second column & $\mathrm{3^{rd}}$ column & $\mathrm{4^{th}}$ column
        \tabularnewline\midrule
        1 & 2 & $\delta$   & 4
        \tabularnewline
        1 & 2 & $\epsilon$ & 4
        \tabularnewline\bottomrule
    \end{tabular}
    \label{tab:table5}
\end{table*}


    Usually scientific journals use two-columns text layout, which sometimes
    complicates float placement inside text. This is because not all figures can
    be made narrow enough to fit into one column of text and if it needs to be
    larger, then we get some nasty overlapping, which is the least we want.

    The ``trick'' there is to use a 'stared' version of the \verb|table|
    environment (ie. \verb|table*|). However, this has some limitations as the
    table float then can be added only to the top of the page. A package
    \verb|stfloats| seems to offer slightly more flexibility over the placement
    as it provides means to put the float on the bottom of the page, however,
    the float still can not be placed in the middle of the page.

    In my opinion these options are the only one which look really good, so I do
    not see any limitations here, it is just it is slightly harder to deal with
    such floats than with simple floats.

    Because I need more text, I will insert some dummy text using a package
    \verb|blindtext| and command \verb|\blindtext|.

    ----------------
    \textbf{Dummy text start}

    \blindtext

    \blindtext

    ----------------
    \textbf{Dummy text finish}
\end{multicols}

The source code for the table \ref{tab:table5} is shown bellow:
\lstinputlisting{tab5.tex}


\end{document}

% Editor configuration:
% vim: tw=80:spell:spelllang=en_gb

