% Clever thing to minimize the files I need to edit to change how the files are
% looking. The following one line comment is to trick VIM-LaTeX
% \documentclass{article}
%documentclass{scrartcl}
\KOMAoptions{
    fontsize=10pt
}
\setkomafont{pagenumber}{\bfseries\upshape\oldstylenums}
\renewcommand{\titlefont}{\rm\bfseries\LARGE}
\usepackage{
    ifxetex, 
    ifdraft,
    ifthen
}

\ifxetex
  \usepackage{fontspec}
  \usepackage{xunicode}
  \defaultfontfeatures{Mapping=tex-text} % To support LaTeX quoting style
  \setromanfont{Gentium}
\else
  \usepackage[utf8]{inputenc}
  \usepackage[T1]{fontenc}
  \usepackage{lmodern,textcomp}
\fi

\usepackage{
%    standalone,
%    lastpage,
    geometry,
    scrpage2,
%    setspace,
    amsmath,
    caption,
    calc,
%    floatrow,
}
\usepackage{
%    xcolor,
    graphicx,
    tikz,
    chemfig
}
\usepackage[final]{listings}
\usepackage[version=3]{mhchem}
\usepackage[update,verbose=false]{epstopdf}
\usepackage[colorinlistoftodos,obeyDraft]{todonotes}
\usepackage{hyperref}


% Set the geometry
\geometry{
    paper = a4paper,
    top=3cm,
    bottom=4cm,
    footskip=1cm,
    marginparwidth=3.5cm,
    headsep=1cm
}
\ifoptiondraft{
\geometry{inner=1.5cm, outer=4cm}
}{
\geometry{inner=3.0cm, outer=2.5cm}
}

%\onehalfspacing

% Setup hyperref
\hypersetup{
    colorlinks,
    urlcolor=blue,
    breaklinks
}

\usetikzlibrary{
    arrows,
    decorations.pathmorphing,
    backgrounds,
    positioning,
    fit,
    petri
}

% Define collors
\definecolor{myyellow}{HTML}{FFFAC9}
\definecolor{myyellowl}{HTML}{FFFBDD}

% Define lstlisting env
\lstset{
    language=[LaTeX]TeX,
    backgroundcolor=\color{myyellowl},
    numbers=left,
    numberstyle=\footnotesize,
    breaklines=true,
    breakatwhitespace=true,
    print=true
}

% Renew 2 styles
\renewpagestyle{plain}{{}{}{}}{{}{}
{\hfill\pagemark{}}}
\renewpagestyle{headings}{{}{}{}}{{}{}
{\hfill\pagemark{}}}

% Set the headings page style
\pagestyle{headings}

% Text mode commands
\newcommand{\uurl}[2]{\href{#1}{#2}\footnote{The URL is \url{#1}}}
\newcommand{\ftype}[1]{\texttt{.#1}}
\newcommand{\fname}[2]{\texttt{#1.#2}}
\newcommand{\pkg}[1]{\texttt{#1}}
\newcommand{\env}[1]{\texttt{#1}}
\newcommand{\cmd}[1]{\texttt{\textbackslash{}#1}}
\newcommand{\usepkg}[2]{
    \texttt{\textbackslash{}usepackage%
    \ifthenelse{\equal{#2}{}}{}{[#2]}\{#1\}}}
\newcommand{\comment}[1]{}

% New commands which ease the work. Units and relative uncertainties
\newcommand{\unit}[1]{\ensuremath{\, \mathrm{#1}}}
\newcommand{\rel}[1]{\ensuremath{ \cfrac{\Delta #1}{#1}}}
\newcommand{\eten}[1]{\ensuremath{ \times 10^{#1}}}
\newcommand{\DP}[2]{\ensuremath{\cfrac{\partial #1}{\partial #2}}}
\newcommand{\DD}[2]{\ensuremath{\cfrac{\mathrm{d} #1}{\mathrm{d} #2}}}
\newcommand{\dd}[1]{\ensuremath{\mathrm{d}#1}}

% Alter some LaTeX defaults for better treatment of figures:
    % See p.105 of "TeX Unbound" for suggested values.
    % See pp. 199-200 of Lamport's "LaTeX" book for details.
    %   General parameters, for ALL pages:
    \renewcommand{\topfraction}{0.9}    % max fraction of floats at top
    \renewcommand{\bottomfraction}{0.8} % max fraction of floats at bottom
    %   Parameters for TEXT pages (not float pages):
    \setcounter{topnumber}{2}
    \setcounter{bottomnumber}{2}
    \setcounter{totalnumber}{4}     % 2 may work better
    \setcounter{dbltopnumber}{2}    % for 2-column pages
    \renewcommand{\dbltopfraction}{0.9} % fit big float above 2-col. text
    \renewcommand{\textfraction}{0.07}  % allow minimal text w. figs
    %   Parameters for FLOAT pages (not text pages):
    \renewcommand{\floatpagefraction}{0.7}      % require fuller float pages
    % N.B.: floatpagefraction MUST be less than topfraction !!
    \renewcommand{\dblfloatpagefraction}{0.7}   % require fuller float pages

\captionsetup{
    format          = plain,        %
    labelformat     = simple,       %
    labelsep        = period,       %
    justification   = default,      %
    font            = default,      %
    labelfont       = {bf,sf},      %
    textfont        = default,      %
    margin          = 0pt,          %
    indention       = 0pt,          %
    parindent       = 0pt,          %
    hangindent      = 0pt,          %
    singlelinecheck = false         %
}

\renewcommand{\thefigure}{\oldstylenums{\arabic{figure}}}

\setatomsep{5mm}
\setbondoffset{.5mm}
\setcrambond{2.5pt}{1pt}{2pt}
\setbondstyle{thick}
\renewcommand*\printatom[1]{{\footnotesize\ensuremath{\mathsf{#1}}}}


% Custom packages
\newcommand\cyrtext[1]{{\fontencoding{T2A}\selectfont #1}}

% Version of the XeTeX logo that doesn't depend on Ǝ being available in the font
% NB: not suitable for use in italic!
% Requires XeTeX 0.7 or later
\def\reflect#1{{\setbox0=\hbox{#1}\rlap{\kern0.5\wd0
  \special{x:gsave}\special{x:scale -1 1}}\box0 \special{x:grestore}}}
\def\XeTeX{\leavevmode
  \setbox0=\hbox{X\lower.5ex\hbox{\kern-.15em\reflect{E}}\kern-.1667em \TeX}%
  \dp0=0pt\ht0=0pt\box0 }

\title{Fonts in \LaTeX{}}
\author{Ignas Anikevicius}

\begin{document}

\maketitle

\begin{abstract}
    Being able to customize fonts in your \verb|.tex| document can be tricky at
    times, or as easy as loading a package or two. In this tutorial I will talk
    through various ways on how to accomplish it.
\end{abstract}

\tableofcontents

\section{UTF8 usage}

This can be accomplished by several ways. Older \LaTeX{} compilers do not
assume, that the input will be given in Unicode character, and thus, if you try
to type something like 
ąžuolas or 
\cyrtext{добридень} or 
clichè, then it will give you
errors unless the file itself is encoded in utf8 and the compiler `knows' that
the input encoding is utf8 and that can be easily achieved by using package
called \verb|inputenc| and it should be executed as follows:
\begin{lstlisting}
\usepackage[utf8]{intputenc} % for utf8
\end{lstlisting}

If you need to produce \LaTeX{} documents with short Cyrillic snippets, then it
would be best to check out this
\uurl{http://win.ua.ac.be/~nschloe/content/latex-and-cyrillic-text-snippets}{website}
as it explains most of the nuances about different font handling in different
char-sets.

For a complete guide how to set up your \LaTeX{} document for various languages
there is a wonderful list on the
\uurl{https://secure.wikimedia.org/wikibooks/en/wiki/LaTeX/Internationalization}{the
\LaTeX{} wikibook}.

\section{The Font Catalogue}

There is a very good resource for fonts, which outlines how to make different
fonts working and it includes notes whether the mathematics mode is supported
with those fonts. 
%
The resource is located at \href{http://www.tug.dk/FontCatalogue/}{\LaTeX\ font
catalogue} website and you will find the information you search.

As for fonts without the support for mathematical symbols, I would advice you
\textbf{against} using them unless you do not use any mathematical  typesetting
in your documents (e.g. this is good for letters and other text-only documents).
%
However, if you need to do that, please consult professionals, some geek-written
blogs or just people who are more experienced in that field.

Now let's talk about commands involved setting different fonts. 
%
Usually you should declare in your preamble, that you will use Type1 fonts,
which look better and most of the scientific journals require them. 
%
Here is how you do it:
\begin{lstlisting}
\usepackage[T1]{fontenc}
\end{lstlisting}

And then you put a package responsible for defining the font:
\begin{description}
    \item[Computer Modern fonts] They are default, so you do not have anything
        to specify, however, there exist some packages which provide some
        variations of those fonts.
        
        For example, if you want slightly enhanced version
        \footnote{See this
        \href{http://www.tug.dk/FontCatalogue/lmodern/}{web page}
        (\url{http://www.tug.dk/FontCatalogue/lmodern/})}
        just put \verb|\usepackage{lmodern}| after loading \verb|fontenc|
        package.
    \item[Times fonts] Just put \verb|\usepackage{mathptmx}| after the previous
        command;
    \item[Palatino] Alternatively for palatino fonts, just write
        \verb|\usepackage{mathpazo}| command;
\end{description}

If you want other types of fonts \emph{with} `math' support, then visit the
\TeX{} font catalogue.

\section{\XeTeX{} or Lua\TeX{} usage}

There are also different compilers than the standard \verb|latex|,
\verb|pslatex| or \verb|pdflatex| (NB the first two are deprecated and only the
third compiler is still developed on). 
%
The alternatives offer different feature set and might be more suitable for
nowadays requirements, such as easier font set up or proper Unicode support
built into the programs themselves. 
%
\XeTeX{} \TeX flavour provide its own 2 compilers named \verb|xetex| and \verb|xelatex|,
whereas \LaTeX{} people have chosen \verb|lualatex| as the successor of the
\verb|pdflatex| compiler and it provides an easy way to include Lua programming
scripts \emph{inside} your \LaTeX{} document to do various tasks.
 
 
 
The Unicode problem in these alternative compilers is solved as they \emph{do}
assume that the input is encoded in Unicode formats and there are no errors when
you enter Unicode text.
%
What is more, they have much better support for changing fonts and \verb|xetex|
and \verb|xelatex| compilers can even utilise TrueType or other modern font
technologies.
%
On the other hand this `flexibility' might cause more additional issues, as
the 'math' mode support for common fonts is usually very poor.
%
That is why you should only use \verb|xelatex| or \verb|xetex| if you do not
need mathematical expressions in your document and you \emph{require} a
different font.
%
There is another issue though --- the alternatives are not as time-proven and
are still under heavy development and they do not integrate as well with older
packages (e.g.  \verb|epstopdf|), which might be unacceptable in some
situations.

\end{document}

% Editor configuration:
% vim: tw=80:spell:spelllang=en_gb
