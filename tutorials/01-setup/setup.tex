% Clever thing to minimize the files I need to edit to change how the files are
% looking. The following one line comment is to trick VIM-LaTeX
 \documentclass[
    draft
]{scrartcl}
%documentclass{scrartcl}
\KOMAoptions{
    fontsize=10pt
}
\setkomafont{pagenumber}{\bfseries\upshape\oldstylenums}
\renewcommand{\titlefont}{\rm\bfseries\LARGE}
\usepackage{
    ifxetex, 
    ifdraft,
    ifthen
}

\ifxetex
  \usepackage{fontspec}
  \usepackage{xunicode}
  \defaultfontfeatures{Mapping=tex-text} % To support LaTeX quoting style
  \setromanfont{Gentium}
\else
  \usepackage[utf8]{inputenc}
  \usepackage[T1]{fontenc}
  \usepackage{lmodern,textcomp}
\fi

\usepackage{
%    standalone,
%    lastpage,
    geometry,
    scrpage2,
%    setspace,
    amsmath,
    caption,
    calc,
%    floatrow,
}
\usepackage{
%    xcolor,
    graphicx,
    tikz,
    chemfig
}
\usepackage[final]{listings}
\usepackage[version=3]{mhchem}
\usepackage[update,verbose=false]{epstopdf}
\usepackage[colorinlistoftodos,obeyDraft]{todonotes}
\usepackage{hyperref}


% Set the geometry
\geometry{
    paper = a4paper,
    top=3cm,
    bottom=4cm,
    footskip=1cm,
    marginparwidth=3.5cm,
    headsep=1cm
}
\ifoptiondraft{
\geometry{inner=1.5cm, outer=4cm}
}{
\geometry{inner=3.0cm, outer=2.5cm}
}

%\onehalfspacing

% Setup hyperref
\hypersetup{
    colorlinks,
    urlcolor=blue,
    breaklinks
}

\usetikzlibrary{
    arrows,
    decorations.pathmorphing,
    backgrounds,
    positioning,
    fit,
    petri
}

% Define collors
\definecolor{myyellow}{HTML}{FFFAC9}
\definecolor{myyellowl}{HTML}{FFFBDD}

% Define lstlisting env
\lstset{
    language=[LaTeX]TeX,
    backgroundcolor=\color{myyellowl},
    numbers=left,
    numberstyle=\footnotesize,
    breaklines=true,
    breakatwhitespace=true,
    print=true
}

% Renew 2 styles
\renewpagestyle{plain}{{}{}{}}{{}{}
{\hfill\pagemark{}}}
\renewpagestyle{headings}{{}{}{}}{{}{}
{\hfill\pagemark{}}}

% Set the headings page style
\pagestyle{headings}

% Text mode commands
\newcommand{\uurl}[2]{\href{#1}{#2}\footnote{The URL is \url{#1}}}
\newcommand{\ftype}[1]{\texttt{.#1}}
\newcommand{\fname}[2]{\texttt{#1.#2}}
\newcommand{\pkg}[1]{\texttt{#1}}
\newcommand{\env}[1]{\texttt{#1}}
\newcommand{\cmd}[1]{\texttt{\textbackslash{}#1}}
\newcommand{\usepkg}[2]{
    \texttt{\textbackslash{}usepackage%
    \ifthenelse{\equal{#2}{}}{}{[#2]}\{#1\}}}
\newcommand{\comment}[1]{}

% New commands which ease the work. Units and relative uncertainties
\newcommand{\unit}[1]{\ensuremath{\, \mathrm{#1}}}
\newcommand{\rel}[1]{\ensuremath{ \cfrac{\Delta #1}{#1}}}
\newcommand{\eten}[1]{\ensuremath{ \times 10^{#1}}}
\newcommand{\DP}[2]{\ensuremath{\cfrac{\partial #1}{\partial #2}}}
\newcommand{\DD}[2]{\ensuremath{\cfrac{\mathrm{d} #1}{\mathrm{d} #2}}}
\newcommand{\dd}[1]{\ensuremath{\mathrm{d}#1}}

% Alter some LaTeX defaults for better treatment of figures:
    % See p.105 of "TeX Unbound" for suggested values.
    % See pp. 199-200 of Lamport's "LaTeX" book for details.
    %   General parameters, for ALL pages:
    \renewcommand{\topfraction}{0.9}    % max fraction of floats at top
    \renewcommand{\bottomfraction}{0.8} % max fraction of floats at bottom
    %   Parameters for TEXT pages (not float pages):
    \setcounter{topnumber}{2}
    \setcounter{bottomnumber}{2}
    \setcounter{totalnumber}{4}     % 2 may work better
    \setcounter{dbltopnumber}{2}    % for 2-column pages
    \renewcommand{\dbltopfraction}{0.9} % fit big float above 2-col. text
    \renewcommand{\textfraction}{0.07}  % allow minimal text w. figs
    %   Parameters for FLOAT pages (not text pages):
    \renewcommand{\floatpagefraction}{0.7}      % require fuller float pages
    % N.B.: floatpagefraction MUST be less than topfraction !!
    \renewcommand{\dblfloatpagefraction}{0.7}   % require fuller float pages

\captionsetup{
    format          = plain,        %
    labelformat     = simple,       %
    labelsep        = period,       %
    justification   = default,      %
    font            = default,      %
    labelfont       = {bf,sf},      %
    textfont        = default,      %
    margin          = 0pt,          %
    indention       = 0pt,          %
    parindent       = 0pt,          %
    hangindent      = 0pt,          %
    singlelinecheck = false         %
}

\renewcommand{\thefigure}{\oldstylenums{\arabic{figure}}}

\setatomsep{5mm}
\setbondoffset{.5mm}
\setcrambond{2.5pt}{1pt}{2pt}
\setbondstyle{thick}
\renewcommand*\printatom[1]{{\footnotesize\ensuremath{\mathsf{#1}}}}


% Custom stuff
\usepackage{perpage}
\MakePerPage{footnote}
\lstset{language=bash,numbers=none}
\newcommand{\MiKTeX}{MiK\TeX}
\newcommand{\MacTeX}{Mac\TeX}
\newcommand{\stress}[1]{\textbf{\uppercase{#1}}}

% Title and similar stuff
\title{How to set up your computer to start using \LaTeX{}}
\author{Ignas Anikevičius}

\begin{document}

\maketitle
\tableofcontents
\listoftodos{\vskip 1em}

%
Thank you for your interest in \LaTeX{} typesetting system and this article
    will help you to get you ready for starting to use \LaTeX{} on your
    computer.

%
Although I would like to write a continuous text on how to install
    everything on different kinds of OSes (Operating Systems), I believe, that
    it is not necessary to duplicate any content, if it can be found in a better
    shape elsewhere. 
%
Therefore, I suggest you reading chapters of the book called \LaTeX{} hosted on
    the website called wikibooks.org. 
%
You can find an on-line version of the
    \uurl{https://secure.wikimedia.org/wikibooks/en/wiki/LaTeX}{book} or the
    \uurl{http://upload.wikimedia.org/wikipedia/commons/2/2d/LaTeX.pdf}{PDF}
    version of it, which I think is much more suitable for reading or printing.

%
The list of the needed software is already there and if you feel very
    comfortable with your system, no specific directions should be necessary.

% ----------------------------------------------------------------------
\comment{
\section{Software from the Department of Chemistry}
% ----------------------------------------------------------------------

%
Computer Office is already providing images for deploying the whole OS and
    necessary software for Chemistry Department members. 
%
As far as I was informed, there are images for Linux and Windows systems.
%
For Macs, there might be customized installers as well available on
    somelink.
\todo{Ask computer officers for a link to the easy mac installers.}
}

% ----------------------------------------------------------------------
\section{\LaTeX{} distribution installation}
% ----------------------------------------------------------------------

%
There are several \LaTeX{} distributions which you can choose from and what
    choice you make might influence your experience with \LaTeX{} a lot.
%
\begin{description}
    \item[``\TeX{} Live''] distribution is available for Linux/Mac/Windows, but
            is usually the preferred option on Linux machines.
        %
    \item[``\MacTeX{}''] distribution is mainly a compilation of the
            ``\TeX{} Live'' software specifically tailored for machines running
            OS X. 
        %
        This should be the preferred way of installing a \LaTeX{} distribution
            on a Mac.
        %
    \item[``pro\TeX{}t''] distribution is available for Windows machines only and
            it is designed so that everything is done for the user
            automatically. 
        %
        It is based on \MiKTeX{} -- another \TeX{} distribution for Windows OS,
            but it offers a much more integrated installation and is very suited
            for beginners as it contains the full \TeX{} distribution
            installation, a good editor and GhostScript installation (program
            for \ftype{ps} files) as well.
        %
        This should be the preferred distribution on Windows OS if you are a
        beginner, otherwise please research ways of how to install a
        \TeX{} Live distribution as it seems to be more secure and faster in
        some occasions. 
        %
\end{description}

\subsection{Notes for Linux users}

Use your Linux Distribution package manager whenever you can and install
``\TeX{} Live'' only from there. If you do not know how to do it, please refer to
your Distribution Wikipedia and search it for 'LaTeX' or 'TeX Live'.

Here is a list of most popular distributions and links to their Wikipedia Pages:

\begin{description}
    \item[.deb based] 
        \uurl{http://wiki.debian.org/Latex}{Debian},
        \uurl{https://help.ubuntu.com/community/LaTeX}{Ubuntu},
        and for distributions, which are derived from these two, the same wiki
        pages can be used. However, for full installation of \TeX{} Live you can try
        these terminal commands (issue them as root):
\begin{lstlisting}
aptitude update
aptitude install texlive-full
\end{lstlisting}

    \item[.rpm based]
        For Fedora, RedHat, CentOS and openSUSE distributions, use your package
        manager and install the full \TeX{} Live distribution. The following
        command executed in terminal as root should work:
\begin{lstlisting}
yum install texlive-full
\end{lstlisting}

    \item[ArchLinux and derivatives]
        The following instruction should work on ArchLinux and Chackra
        distributions. Both use pacman as their packages manager, so the
        following commands executed as root user will suffice:
\begin{lstlisting}
pacman -S texlive-most texlive-lang
\end{lstlisting}
        NB the second package texlive-lang is for the different languages
        support and if you use only English language, then you are free to
        install only the \verb|texlive-most| package

        Archlinux has a very good wiki article:
        \url{https://wiki.archlinux.org/index.php/TeX_Live}.
        
    \item[Gentoo and derivatives]
        This applies for Gentoo, Funtoo, Sabayon distributions. Things which
        work will definitely work on the other two distributions, so we will
        analyse only Gentoo.

        For checking the list of functionality for texlive distribution, enter:
\begin{lstlisting}
equery uses texlive
\end{lstlisting}
        and you will get the list of available use flags.
        You will have to enable the needed flags in the
        \verb|/etc/portage/package.use| and then just emerge the packages:
\begin{lstlisting}
emerge -av texlive
\end{lstlisting}
        In order to get a newer version of \TeX{} Live, just unmask the needed
        packages via \verb|/etc/portage/package.accept_keywords|.

        Gentoo has a very good
        \uurl{http://www.gentoo.org/proj/en/tex/texlive-migration-guide.xml}{Wiki}
        page documenting the installation.
    \item[Others]
        Install the \TeX{} Live distribution via your distributions packet
        manager. If you do not know how to do that, ask in the forums on your
        distribution web page.
\end{description}

\subsection{Notes for Mac users}

For easier experience, just install the full \MacTeX{} installation which can be
found on the following \uurl{http://www.tug.org/mactex/}{website}.

\subsection{Notes for Windows users}

%
For easier experience, download pro\TeX{}t installation files from
    \uurl{http://www.tug.org/protext/}{their website}.
%
The installation above will provide you with a \TeX{}nicCenter \LaTeX{} IDE and
    full \TeX{} distribution installation.

%
Another option would be to install either the \MiKTeX{} distribution directly
    instead of the pro\TeX{}t distribution.
%
It can be downloaded from \uurl{http://www.miktex.org/2.9/setup}{this web site}
    and there are two options of installation:
\begin{description}
    \item[Install everything] 
        Although this might be very convenient as
            one will not have to worry about missing packages, but it takes
            space.
        %
        On the other hand, slightly more than 1GB of occupied space on modern
            computers will not make a difference.
        %
    \item[Install a base system] 
        This is the alternative, which would take less
            space. 
        %
        What is more, one can select an option where necessary packages
            could be installed on the fly without any user intervention.
        %
        However, it might need some more configuration afterwards as it seems
            that this method sometimes does not work ``automagically''.
        %
\end{description}

%
And yet the last option would be to install the \TeX{} Live distribution.
%
This might be the preferred way for people who want to have the same environment
    across all Operating Systems they use.
%
What is more, sometimes it may compile the documents faster and it is thought to
    be a more secure distribution.
%
For ways how to install this distribution, please use
    \href{http://www.google.co.uk}{google}.


\section{Editing a \ftype{tex} file}

Mainly there are two choices:
\begin{itemize}
    \item Integrated Development Environment
        \uurl{https://secure.wikimedia.org/wikipedia/en/wiki/Integrated_development_environment}{IDE}
    \item Just a text editor.
\end{itemize}

%
While IDEs generally will provide a user with much more integrated environment,
    this does not necessarily mean, that producing \LaTeX{} documents with an
    IDE is faster. 
%
There are many very powerful text editors, which might have a steep learning
    curve, but once mastered, they are very fast.
%
What is more, some text editors might be better in some tasks than other, so
    there is no such thing as ``the best'' IDE or text editor for \LaTeX{}.

%
The most important projects are mentioned bellow:
\begin{description}
    \item[VIM \& Emacs] 
        VIM is the best editor, in my opinion.
        %
        It is very fast, lightweight and it can be customized a lot.
        %
        Although it has a steep learning curve, it is very rewarding afterwards
            and reading any of the books on VIM would help a lot. 
        
        %
        This being said, everybody admit, that Emacs is also good, and many
            argue that it is better than VIM.
        %
        This has much to do with so-called
        \uurl{http://en.wikipedia.org/wiki/Editor_war}{editor wars}.

        %
        Since both are very advanced editors, you will find that they have very
            powerful \LaTeX{} plug-ins, which might make the work faster than
            with most of the IDEs.
        %
        On the other hand, both editors present a user some difficulties while
            learning the commands as the learning curve is somewhat steep(ish).

    \item[LyX] 
        This is a project, that aims the user to give a word-processor, which
            would use \LaTeX{} internally.
        %
        Although one can achieve really good results with it, technically you do
            not write \LaTeX{} and it will not help you at all with \LaTeX{} if
            you want to learn it.
        %
        However, since it does a lot of automatic things, it might be a very
            good reference tool for searching hints how to achieve some things
            with \LaTeX{} (e.g. searching for symbols, remembering commands). 
        
        %
        That said, I would like to strongly advice you \stress{against using
            this word processor, other than for reference matters!}
        %
        The reason is because publishers do not accept LyX files and once you
            export them to \LaTeX{}, it becomes a mess.
        %
        What is more, it will be easier to collaborate with colleagues if you
            use \LaTeX{}.
        %
        And sometimes, it tries to do more, than you want, or ask it to do and
            then you get errors, and spend so much precious time debugging
            instead of writing your thesis.

        %
    \item[\TeX{}Shop] 
        This is probably the first good \LaTeX{} IDE for Mac, which was highly
            successful when it first came out.
        %
        However, it is still being widely used and most of Mac people will
            suggest it first.

        %
    \item[\TeX{}nicCenter] 
        \TeX{}nicCenter is probably the best choice for Windows users, as it has
            a lot of functionality.
        %
        If you choose this IDE, then you won't have to worry about anything as
            there are a lot of resources on the internet on how to get various
            features working and etc.
        %
        What is also good about it, is that the user has complete control over
            the majority of its settings and it is very easy for sophisticated
            users to get what they want.

        %
    \item[\TeX{}works]
        A cross-platform IDE which was inspired by \textbf{TeXShop}.
        %
        Note, that it is very useful as it comes with every \LaTeX{}
            distribution by default (except for \TeX{} Live).
        %
        Thus you can start using it very quickly.

        %
    \item[\TeX{}Maker]
        A good cross-platform IDE using Qt toolkit.
        %
        This might be an alternative to \TeX{}works, although there are better
            alternatives listed above, so it is here just for completeness.
        %
\end{description}

Other projects, which can be still very well used to achieve good results, but
    are somewhat less popular:
%
\begin{description}
    \item[Kile]
        This is an IDE for Linux, mostly suited for KDE users.
        %
    \item[Geany]
        This is an IDE for Linux, built with the GTK+ widget set and might be
            the primary choice for those people who want an IDE, but don't want
            to install Qt, or have other strong opinions towards the Qt library.
        %
        \todo{Add more detail?}
    \item[Others]
        There are probably more ways to create \ftype{tex} documents than I have
            mentioned here and if you feel, that none of the solutions works for
            you, please research the web and you'll find alternatives.
\end{description}

% ----------------------------------------------------------------------
\section{Bibliography management}
% ----------------------------------------------------------------------

%
Bibliography in \LaTeX{} can be managed in several ways and information on how it
    is done is given in the Bibliography Tutorial.
    \todo{Link to it!}
%
Here I will just flick through different software which can act as a front-end
    to Bib\TeX{}.
%
Those front-ends can be categorized as follows:
%
\begin{description}
    \item[\LaTeX{} IDEs]
        Some \LaTeX{} IDEs are known to be able to manage \ftype{bib} files
            (native Bib\TeX{} format), however, to check whether your is capable
            of doing that, you should refer to the documentation of the software
            you use or just quick-check on web-search engines like
            \uurl{http://www.google.co.uk}{Google},
            \uurl{http://www.bing.co.uk}{Bing} or others.
        %
    \item[Text Editor]
        You can manage the bibliography manually, by using just your text
            editor.
        %
        For this job I would suggest only the more advanced one like VIM or
            Emacs, however, you can create the \ftype{bib} file with anything
            you want, as long as it supports ASCII encoding (which should be the
            case for all text editors).
        %
        The templates can be found on the
            \uurl{https://secure.wikimedia.org/wikibooks/en/wiki/LaTeX/Bibliography_Management\#Standard_templates}{\LaTeX{}
            wikibook}.
        %
    \item[Bib\TeX{} GUIs]
        This is the third category and probably the most suited for the job.
        %
        There are wonderful bibliography managers, which can export the database
            as \ftype{bib} files and, thus, can be easily used with \LaTeX{}.
        %
        The list of software can be found either on 
            \uurl{https://secure.wikimedia.org/wikipedia/en/wiki/Comparison_of_reference_management_software}{Wikipedia} 
            or
            \uurl{https://secure.wikimedia.org/wikibooks/en/wiki/LaTeX/Bibliography_Management\#Helpful_Tools}{\LaTeX{} wikibook}.
        %
\end{description}

% ----------------------------------------------------------------------
\section{PDF viewers}
% ----------------------------------------------------------------------

%
Good PDF viewers are different across different platforms.
%
I believe, that you might want say, that Adobe's PDF viewer is very good, but
    the truth is that it is slow and not as reliable as others, so please be
    sure, that you check, or at least are aware of alternatives.

%
A much better alternative might look \textbf{Foxit} PDF reader, which is
    available for both Linux and Windows operating systems.
%
However, by no means it is the best solution and one should research a bit
    before settling down with the most appealing PDF viewer.

%
\subsection{On Linux}

%
Linux users have a huge variety of PDF viewers to select from.
%
One should search distribution's repositories, but just to mention a few:
\begin{description}
    \item[IDE's viewer]
        If you will be using a fully-fledged IDE for \LaTeX{} (like \TeX{}works
            or similar), then it is very likely that you will not need another
            PDF viewer for monitoring the changes to the \ftype{tex} file.
        %
    \item[Evince]
        Default PDF viewer for GNOME.
        %
        Supports Sync\TeX{}, which is very useful for improving the user
            experience and document creating.
        %
    \item[Okular] 
        Default PDF viewer for KDE.
        %
        Like Evince, supports Sync\TeX{} and might be more feature-full than the
            former.
        %
    \item[mupdf]
        This is a PDF rendering library with a very minimalistic viewer.
        %
        Although it might be very minimalistic, it is very fast and reliable.
        %
        This library is cross-platform, so the viewer might be available on
            other platforms as well.
        %
    \item[zathura and apvlv]
        These are minimalistic and very fast VIM-like PDF viewers suited for
            geeks.
        %
\end{description}

%
\subsection{On Mac}

Your IDE is very likely to be bundled with some custom PDF viewer and thus you
    might not need an extra one.
%
However if you need Sync\TeX{} support, which improves the user experience a
    lot, keep in mind, that not all IDE PDF viewer support it.
%
In case you wanted the support in a external viewer, you should definitely check
    out \textbf{Skim} PDF viewer.

%
A good choice just to preview files is the native program called
    \textbf{Preview}.

%
\subsection{On Windows}

The best choice would be a \textbf{Sumatra} PDF viewer.
%
It is based on very stable and fast MuPDF rendering library and it supports lots
    of other features, of which the most important one is Sync\TeX{}.
%
This viewer launches immediately, which is very useful if you close and open the
    PDF files often.

%
Other alternatives either need to be paid for or they are not as
    reliable/feature-complete as the \textbf{Sumatra} PDF viewer.

\end{document}

% Editor configuration:
% vim: tw=80:spell:spelllang=en_gb
