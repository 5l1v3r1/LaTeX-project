% Clever thing to minimize the files I need to edit to change how the files are
% looking. The following one line comment is to trick VIM-LaTeX
\documentclass[
    draft
]{scrartcl}
%documentclass{scrartcl}
\KOMAoptions{
    fontsize=10pt
}
\setkomafont{pagenumber}{\bfseries\upshape\oldstylenums}
\renewcommand{\titlefont}{\rm\bfseries\LARGE}
\usepackage{
    ifxetex, 
    ifdraft,
    ifthen
}

\ifxetex
  \usepackage{fontspec}
  \usepackage{xunicode}
  \defaultfontfeatures{Mapping=tex-text} % To support LaTeX quoting style
  \setromanfont{Gentium}
\else
  \usepackage[utf8]{inputenc}
  \usepackage[T1]{fontenc}
  \usepackage{lmodern,textcomp}
\fi

\usepackage{
%    standalone,
%    lastpage,
    geometry,
    scrpage2,
%    setspace,
    amsmath,
    caption,
    calc,
%    floatrow,
}
\usepackage{
%    xcolor,
    graphicx,
    tikz,
    chemfig
}
\usepackage[final]{listings}
\usepackage[version=3]{mhchem}
\usepackage[update,verbose=false]{epstopdf}
\usepackage[colorinlistoftodos,obeyDraft]{todonotes}
\usepackage{hyperref}


% Set the geometry
\geometry{
    paper = a4paper,
    top=3cm,
    bottom=4cm,
    footskip=1cm,
    marginparwidth=3.5cm,
    headsep=1cm
}
\ifoptiondraft{
\geometry{inner=1.5cm, outer=4cm}
}{
\geometry{inner=3.0cm, outer=2.5cm}
}

%\onehalfspacing

% Setup hyperref
\hypersetup{
    colorlinks,
    urlcolor=blue,
    breaklinks
}

\usetikzlibrary{
    arrows,
    decorations.pathmorphing,
    backgrounds,
    positioning,
    fit,
    petri
}

% Define collors
\definecolor{myyellow}{HTML}{FFFAC9}
\definecolor{myyellowl}{HTML}{FFFBDD}

% Define lstlisting env
\lstset{
    language=[LaTeX]TeX,
    backgroundcolor=\color{myyellowl},
    numbers=left,
    numberstyle=\footnotesize,
    breaklines=true,
    breakatwhitespace=true,
    print=true
}

% Renew 2 styles
\renewpagestyle{plain}{{}{}{}}{{}{}
{\hfill\pagemark{}}}
\renewpagestyle{headings}{{}{}{}}{{}{}
{\hfill\pagemark{}}}

% Set the headings page style
\pagestyle{headings}

% Text mode commands
\newcommand{\uurl}[2]{\href{#1}{#2}\footnote{The URL is \url{#1}}}
\newcommand{\ftype}[1]{\texttt{.#1}}
\newcommand{\fname}[2]{\texttt{#1.#2}}
\newcommand{\pkg}[1]{\texttt{#1}}
\newcommand{\env}[1]{\texttt{#1}}
\newcommand{\cmd}[1]{\texttt{\textbackslash{}#1}}
\newcommand{\usepkg}[2]{
    \texttt{\textbackslash{}usepackage%
    \ifthenelse{\equal{#2}{}}{}{[#2]}\{#1\}}}
\newcommand{\comment}[1]{}

% New commands which ease the work. Units and relative uncertainties
\newcommand{\unit}[1]{\ensuremath{\, \mathrm{#1}}}
\newcommand{\rel}[1]{\ensuremath{ \cfrac{\Delta #1}{#1}}}
\newcommand{\eten}[1]{\ensuremath{ \times 10^{#1}}}
\newcommand{\DP}[2]{\ensuremath{\cfrac{\partial #1}{\partial #2}}}
\newcommand{\DD}[2]{\ensuremath{\cfrac{\mathrm{d} #1}{\mathrm{d} #2}}}
\newcommand{\dd}[1]{\ensuremath{\mathrm{d}#1}}

% Alter some LaTeX defaults for better treatment of figures:
    % See p.105 of "TeX Unbound" for suggested values.
    % See pp. 199-200 of Lamport's "LaTeX" book for details.
    %   General parameters, for ALL pages:
    \renewcommand{\topfraction}{0.9}    % max fraction of floats at top
    \renewcommand{\bottomfraction}{0.8} % max fraction of floats at bottom
    %   Parameters for TEXT pages (not float pages):
    \setcounter{topnumber}{2}
    \setcounter{bottomnumber}{2}
    \setcounter{totalnumber}{4}     % 2 may work better
    \setcounter{dbltopnumber}{2}    % for 2-column pages
    \renewcommand{\dbltopfraction}{0.9} % fit big float above 2-col. text
    \renewcommand{\textfraction}{0.07}  % allow minimal text w. figs
    %   Parameters for FLOAT pages (not text pages):
    \renewcommand{\floatpagefraction}{0.7}      % require fuller float pages
    % N.B.: floatpagefraction MUST be less than topfraction !!
    \renewcommand{\dblfloatpagefraction}{0.7}   % require fuller float pages

\captionsetup{
    format          = plain,        %
    labelformat     = simple,       %
    labelsep        = period,       %
    justification   = default,      %
    font            = default,      %
    labelfont       = {bf,sf},      %
    textfont        = default,      %
    margin          = 0pt,          %
    indention       = 0pt,          %
    parindent       = 0pt,          %
    hangindent      = 0pt,          %
    singlelinecheck = false         %
}

\renewcommand{\thefigure}{\oldstylenums{\arabic{figure}}}

\setatomsep{5mm}
\setbondoffset{.5mm}
\setcrambond{2.5pt}{1pt}{2pt}
\setbondstyle{thick}
\renewcommand*\printatom[1]{{\footnotesize\ensuremath{\mathsf{#1}}}}


% Custom packages
%\usepackage{natbib}
\usepackage[printonlyused]{acronym}

\title{Cross-referencing, Producing Citations and Composing Bibliography in \LaTeX{}}
\author{Ignas Anikevicius}
\date{}

\begin{document}

\maketitle
\tableofcontents
\listoftodos

Bibliography can be very easily managed in \LaTeX\ and this tutorial will
quickly explain how to get good results in relatively short time.  There are
mainly two frameworks which can be used --- \verb|BibTeX| and the
\verb|bibliography| environment. The latter might not be the most convenient
one, and it will not be talked about much. What you do there is you enter
the whole bibliography in the \verb|bibliography| environment. Previously
mentioned two choices are superior to this as the bibliography database can
be edited with external programs and thus the productivity might increase.

If you feel that you want to read more about various bibliography formats,
you can read it on the LaTeX
\uurl{https://secure.wikimedia.org/wikibooks/en/wiki/LaTeX/Bibliography_Management}{wikibook}
.\cite{latexwikibook:bibliography}


\section{Creating the Database and Bib\TeX\ usage}

Before citing any authors one would need to create a database, which would
consist of all things the author needs to cite. 
%
This might be everything, from articles to scientific blogs or something
similar. 
%
Moreover, one might have different bibliography databases for different things
(e.g. one for articles, one for books etc.). 
%
It does not matter in how many files do you have the bibliography list split,
\LaTeX\ will load all the files and you won't have to worry. 
%
This can be done with relatively easy commands shown in the listing
\ref{lst:egbib} (if you are using Bib\TeX\ then these commands are inserted in
the text, where you want the reference list to appear).
%
\begin{lstlisting}[caption={Example code to get BibTeX working},label=lst:egbib]
\bibliographystyle{plain}       % Load the plain style
\bibliography{file1, file2}     % Load the file1, file2 which are in
                                % the same directory as the .tex
                                % file.
\end{lstlisting}



And now we need to know how to create the database files themselves, namely,
\verb|file1| and \verb|file2| files. 
%
These can be created either with a simple text editor while using templates from
this
\uurl{https://secure.wikimedia.org/wikibooks/en/wiki/LaTeX/Bibliography_Management\#Standard_templates}{link}
where the way multiple authors are handled is described slightly above in the
web page.



The bibliography can be created with software having a \ac{gui}. 
%
This might be a much easier way for Windows and Mac users as the \ac{oses} tend
to focus on GUI software rather then \ac{cli} software.
%
The list of supposedly good software for managing bibliography goes bellow:
%
\begin{description}
    \item[JabRef] This is a very good program written in JAVA, which means that
        it will work on all systems in the same way. So this can be used by
        \textbf{Linux}, \textbf{Mac} and \textbf{Windows} users.
        
    \item[BibDesk] This is a quite good software package for \textbf{Mac} users
        which may be an alternative for the \textbf{JabRef}.
\end{description}


% ---------------------------------------------------------------------
\section{How to make citations}
% ---------------------------------------------------------------------


Since we have a database, from which we can get the citations and the
information about the sources, we should now learn how to make the citations.
%
Basically I can cite a source with a simple command
\verb|\cite{Knut:LaTeX1977}|, which will give me a citations. 
%
However, its appearance is determined by the style of the bibliography. 
%
The style can be altered via \LaTeX\ packages, and it should be done as the last
thing in your document. 
%
All the stiling and formatting should be taken care off at the last stage of
your work. 
%
Just for the time being, let's suppose, that we are using the \verb{plain{ style
which could be invoked as shown in the listing \ref{lst:egbib}.
%
The \verb{\cite{ command would just give you a bracketed citation like ``[1]'',
or ``[2,3,4]'', or even ``[2-4]''. 
%
For the later two examples one should use \verb|\cite{source1,source2,source3}|
structure to get the formating as above.



If you want some source to appear in the bibliography list, but do not have any
particular reason to cite it, just issue \verb|\nocite{somesource}| command.



If you want several sources to appear under the same number (eg. ``[7]''), but
in the bibliography list it to appear as a itemized list (ie.
7.a)\ldots b)\ldots c)\ldots), then you should use the \verb|mciteplus| package.
%
Its documentation can be found following this
\uurl{http://mirror.ctan.org/macros/latex/contrib/mciteplus/mciteplus_doc.pdf}{link},
but the same functionality is provided by the \verb|achemso package|, so if you
use achemso you \textbf{do not} need to include the \verb|mciteplus| package.
%
Once you have the necessary packages loaded, you can just issue
\verb|\cite{source1, *source2, *source3}| to get the desired effect. 
%
However there are several limitations:
%
\begin{itemize}
    \item \verb|\cite{source1, *source2, source3}| will give you an error as the
        \verb|source3| is not preceded with a ``*'' symbol.
    \item if you use \verb|\cite{source1, *source2, *source3}| in the text, the
        next time you want to cite the source, remember to use
\begin{lstlisting}
\cite{source1, *source2, *source3}
\end{lstlisting}
        or
\begin{lstlisting}
\cite{source1, *source2}
\end{lstlisting}
        \textbf{but not}
\begin{lstlisting}
\cite{source1, source2}
\end{lstlisting}
        as it will also give you an error.
    \item Other limitations can be found talked about in the documentation of
        the package.
\end{itemize}


% ---------------------------------------------------------------------
\section{The cite package usage}
% ---------------------------------------------------------------------


If you want to gain more control over how the citations appear in the text, you
can use the \verb|cite| package. 
%
Like all packages, it is invoked with \verb|\usepackage[]{cite}| written in the
preamble where the options to the package go between the ``[]'' brackets.
%
It would worth reading about the options on your-own in the documentation of the
package which can be found through
\uurl{http://mirrors.ctan.org/macros/latex/contrib/cite/cite.pdf}{this link}.
%
Please, read it before you use it!

% ---------------------------------------------------------------------
\section{The natbib package usage}
% ---------------------------------------------------------------------

The \verb|natbib| package can give you more styles of citations and it provides
eveon more extensibility than the \verb|cite| package. 
%
If you feel, that the \verb|cite| package does not provide what you want, please
try this package and please read about it on
\uurl{https://secure.wikimedia.org/wikibooks/en/wiki/LaTeX/Bibliography_Management\#Natbib}{this source}.
%
The documentation can be found
\uurl{ftp://ftp.tex.ac.uk/tex-archive/macros/latex/contrib/natbib/natbib.pdf}
{here}, whereas a useful reference-sheet-web-page can be found
\uurl{http://merkel.zoneo.net/Latex/natbib.php}{here}.

% ---------------------------------------------------------------------
\section{The achemso packge usage}
% ---------------------------------------------------------------------

Achemso package might be very useful for many people who want to get the
citations as superscripts and the Bibliography list as in the \ac{acs} journals.
%
If you use it, be sure, that you do not use natbib, cite or any other packages
which manage bibliographies, as you can experience unexpected results while
theses packages try to redefine the commands.
%
For the usage of this package, please read the 
\uurl{http://www.tex.ac.uk/ctan/biblio/bibtex/contrib/achemso/achemso.pdf}
{documentation},
however, it would be worthwhile to note, that for minimal results, you should
try to include it into you preamble and then define the bibliography style as
\verb|achemso| as follows:
\lstinputlisting[caption={The achemso package
usage},label=lst:achemso1]{./doc-bib1.tex}



If you want to customize the style of the citations, please refer to the package
documentation.

% ---------------------------------------------------------------------
\section{The Bib\LaTeX{} package usage}
% ---------------------------------------------------------------------

Bib\LaTeX{} is thought to be a replacement for natbib and Bib\TeX{} in a long
run and if you whant very quick answer to what exactly is Bib\LaTeX{}, you
should probably check
\href{http://tex.stackexchange.com/questions/8411/what-is-the-difference-between-bibtex-and-biblatex/8416#8416}{this
link} on \verb|stackexchan.com| as it answers the question very well.

Other reasons why you would want to try this package is that it can do the
footnote references, whereas natbib can not. This turns out to be very useful
when you are \textbf{not} allowed to have an extra page for the references. As
for how to do this, follow
\uurl{http://www.charlietanksley.net/philtex/proper-footnote-citations-with-latex/}{this
link}.

Just a note, that most of the packages easing bibliography management, like
natbib, mcite, and others do rely on Bib\TeX{} system and they \textbf{won't}
work with Bib\LaTeX{}, however, it provides all the functionality you would
need and configuration syntax is cleaner. What is more, if you want to create
your own style for displaying references, you do not need to create custom
\verb|.bst| files as all the formatting is done via \LaTeX{} macros. So if you
know \LaTeX{} well, you will be able to come up with good bibliography style
quite easily.

\appendix
\section{Acronyms used in the text}
\begin{acronym}
    \acro{gui}[GUI]{Graphical User Interface}
    \acro{cli}[CLI]{Command Line Interface}
    \acro{os}[OS]{Operating System}
    \acro{oses}[OSes]{Operating Systems}
    \acro{acs}[ACS]{American Chemistry Society}
\end{acronym}

\bibliographystyle{plain}
\bibliography{../tutorial}

\end{document}

% Editor configuration:
% vim: tw=80:spell:spelllang=en_gb
