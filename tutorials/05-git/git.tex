% Clever thing to minimize the files I need to edit to change how the files are
% looking. The following one line comment is to trick VIM-LaTeX
\documentclass[
    draft
]{scrartcl}
%documentclass{scrartcl}
\KOMAoptions{
    fontsize=10pt
}
\setkomafont{pagenumber}{\bfseries\upshape\oldstylenums}
\renewcommand{\titlefont}{\rm\bfseries\LARGE}
\usepackage{
    ifxetex, 
    ifdraft,
    ifthen
}

\ifxetex
  \usepackage{fontspec}
  \usepackage{xunicode}
  \defaultfontfeatures{Mapping=tex-text} % To support LaTeX quoting style
  \setromanfont{Gentium}
\else
  \usepackage[utf8]{inputenc}
  \usepackage[T1]{fontenc}
  \usepackage{lmodern,textcomp}
\fi

\usepackage{
%    standalone,
%    lastpage,
    geometry,
    scrpage2,
%    setspace,
    amsmath,
    caption,
    calc,
%    floatrow,
}
\usepackage{
%    xcolor,
    graphicx,
    tikz,
    chemfig
}
\usepackage[final]{listings}
\usepackage[version=3]{mhchem}
\usepackage[update,verbose=false]{epstopdf}
\usepackage[colorinlistoftodos,obeyDraft]{todonotes}
\usepackage{hyperref}


% Set the geometry
\geometry{
    paper = a4paper,
    top=3cm,
    bottom=4cm,
    footskip=1cm,
    marginparwidth=3.5cm,
    headsep=1cm
}
\ifoptiondraft{
\geometry{inner=1.5cm, outer=4cm}
}{
\geometry{inner=3.0cm, outer=2.5cm}
}

%\onehalfspacing

% Setup hyperref
\hypersetup{
    colorlinks,
    urlcolor=blue,
    breaklinks
}

\usetikzlibrary{
    arrows,
    decorations.pathmorphing,
    backgrounds,
    positioning,
    fit,
    petri
}

% Define collors
\definecolor{myyellow}{HTML}{FFFAC9}
\definecolor{myyellowl}{HTML}{FFFBDD}

% Define lstlisting env
\lstset{
    language=[LaTeX]TeX,
    backgroundcolor=\color{myyellowl},
    numbers=left,
    numberstyle=\footnotesize,
    breaklines=true,
    breakatwhitespace=true,
    print=true
}

% Renew 2 styles
\renewpagestyle{plain}{{}{}{}}{{}{}
{\hfill\pagemark{}}}
\renewpagestyle{headings}{{}{}{}}{{}{}
{\hfill\pagemark{}}}

% Set the headings page style
\pagestyle{headings}

% Text mode commands
\newcommand{\uurl}[2]{\href{#1}{#2}\footnote{The URL is \url{#1}}}
\newcommand{\ftype}[1]{\texttt{.#1}}
\newcommand{\fname}[2]{\texttt{#1.#2}}
\newcommand{\pkg}[1]{\texttt{#1}}
\newcommand{\env}[1]{\texttt{#1}}
\newcommand{\cmd}[1]{\texttt{\textbackslash{}#1}}
\newcommand{\usepkg}[2]{
    \texttt{\textbackslash{}usepackage%
    \ifthenelse{\equal{#2}{}}{}{[#2]}\{#1\}}}
\newcommand{\comment}[1]{}

% New commands which ease the work. Units and relative uncertainties
\newcommand{\unit}[1]{\ensuremath{\, \mathrm{#1}}}
\newcommand{\rel}[1]{\ensuremath{ \cfrac{\Delta #1}{#1}}}
\newcommand{\eten}[1]{\ensuremath{ \times 10^{#1}}}
\newcommand{\DP}[2]{\ensuremath{\cfrac{\partial #1}{\partial #2}}}
\newcommand{\DD}[2]{\ensuremath{\cfrac{\mathrm{d} #1}{\mathrm{d} #2}}}
\newcommand{\dd}[1]{\ensuremath{\mathrm{d}#1}}

% Alter some LaTeX defaults for better treatment of figures:
    % See p.105 of "TeX Unbound" for suggested values.
    % See pp. 199-200 of Lamport's "LaTeX" book for details.
    %   General parameters, for ALL pages:
    \renewcommand{\topfraction}{0.9}    % max fraction of floats at top
    \renewcommand{\bottomfraction}{0.8} % max fraction of floats at bottom
    %   Parameters for TEXT pages (not float pages):
    \setcounter{topnumber}{2}
    \setcounter{bottomnumber}{2}
    \setcounter{totalnumber}{4}     % 2 may work better
    \setcounter{dbltopnumber}{2}    % for 2-column pages
    \renewcommand{\dbltopfraction}{0.9} % fit big float above 2-col. text
    \renewcommand{\textfraction}{0.07}  % allow minimal text w. figs
    %   Parameters for FLOAT pages (not text pages):
    \renewcommand{\floatpagefraction}{0.7}      % require fuller float pages
    % N.B.: floatpagefraction MUST be less than topfraction !!
    \renewcommand{\dblfloatpagefraction}{0.7}   % require fuller float pages

\captionsetup{
    format          = plain,        %
    labelformat     = simple,       %
    labelsep        = period,       %
    justification   = default,      %
    font            = default,      %
    labelfont       = {bf,sf},      %
    textfont        = default,      %
    margin          = 0pt,          %
    indention       = 0pt,          %
    parindent       = 0pt,          %
    hangindent      = 0pt,          %
    singlelinecheck = false         %
}

\renewcommand{\thefigure}{\oldstylenums{\arabic{figure}}}

\setatomsep{5mm}
\setbondoffset{.5mm}
\setcrambond{2.5pt}{1pt}{2pt}
\setbondstyle{thick}
\renewcommand*\printatom[1]{{\footnotesize\ensuremath{\mathsf{#1}}}}


% Custom packages

\newcommand{\TODO}[1]{\todo[inline]{#1}}

\title{How to use GIT version control system to backup and work collaboratively
on the same documents}
\author{Ignas Anikevicius}

\begin{document}

\maketitle

\LaTeX{} is a very good tool for this purpose as the files are created in ASCII
    format and everybody can read it the same way.
%
\tableofcontents
\listoftodos
\clearpage

% ----------------------------------------------------------------------
\section{What is version control system (VCS)?}
% ----------------------------------------------------------------------

There are many models of VCS, but the main idea in all of them remains the same.
%
This is the ability of registering the changes to the files. 
%
A typical workflow can be described as follows:
\begin{enumerate}
    \item Repository is initialized with initial files.
    \item Then a person copies the repository
    \item makes some changes to the files
    \item submits them back to the repository 
    \item clever algorithms detect  which part(s) of the files were altered 
    \item The files are updated and the changes being made are saved
    \item Go back to point 2.
\end{enumerate}

As you see from this description it is clear that the whole history of how the
    files were changing is saved and, thus, they can be restored to any previous
    state.
%
In addition to this being such a good back up tool, there are several
    other advantages one should be aware of:
%
\begin{itemize}
    \item It takes much less space than having multiple folders with different
        versions of the files
    \item One can spot what was changed much more easily
    \item People can work on different parts of the file at the same time.
\end{itemize}

%
The last feature is the most useful for \LaTeX{} users once they need to work
    together with someone.

% ----------------------------------------------------------------------
\section{Using GIT and \LaTeX{} together}
% ----------------------------------------------------------------------

\subsection{Introduction on Git}

Git is a fully open source Distributed Version Control System (DVCS) used for
    many major Open Source Software (OSS) projects.
%
This means, that you get an advanced tool totally for free and you have got an
    army of community members around the world from whom you can learn about git
    and find new 'tricks' almost every day.
%
Distributed VCS from a simple VCS differs in one aspect which makes the former
    much more flexible and simply better than the latter - every user has its
    own copy of the entire repository and can work on it in any possible way.
%
This has some implications:
\begin{itemize}
    \item The user can work off-line and then when he goes back on-line he can
        submit the number of changes to the main repository.
    \item If the main repository goes down or is compromised, you can get the
        source code back to its original state simply because every user has a
        copy of it.
\end{itemize}


%
What is more, git has introduced 'branching' mechanism into VCS.
%
Branching means, that you can create a branch of a repository and then make some
    changes on it and then merge it back whenever you want.
%
Or in the same way you can just abandon it, if you think that the changes you
    made where simply a mistake and you want the previous state of the
    repository back.
%
Or you can even 'cherry-pick' some changes from the branch and merge it back to
    the 'master' branch and discard the others.


\todo{Tell more about Git Pull Push requests.}

This way one can have a very powerful tool to manage all the changes to the
    files on the system.

%
\subsection{Using Git}

Git system is very easy to use if you are used to working in shell (such as Bash
    or Zsh).
%
However, it might require more adjustment for those who have always been using
    Graphic User Interface and are not as comfortable without any buttons or
    icons to press.
%
Luckily there are GUI software, which might help you interface with a Git
    repository very easily:
%
\todo{Add a list containing useful GUI soft for Linux, Mac and Windows OSes.}

%
\subsection{Essential git commands}

%
This section is probably more suited towards people who are using command line
    interfaces to git, but I think all people should be aware of basic git
    commands in order to understand the inner workings better.
%
There is a very good list of commands when you create a new 
\uurl{http://www.github.com}{GitHub} repository:
%
\todo{Describe each of the commands}
%
\begin{description}
    \item[Set up git] Commands for changing system wide settings:
\begin{lstlisting}
git config --global user.name "Your Name which will appear in log messages"
git config --global user.email yourname@email.com
\end{lstlisting}

    \item[Initialize repository] Command for initializing repository:
\begin{lstlisting}
mkdir Project-name
cd Project-name
git init        # This is for initializing the git repository
\end{lstlisting}

    \item[Adding files] Commands for adding various files to already initialized
        repositories:
\begin{lstlisting}
touch README                    # Create an empty text file
git add README                  # Add file to the git repo file-list
git commit -m 'first commit'    # Commit the changes
\end{lstlisting}

    \item[Synchronizing] Adding a remote server and synchronizing the
        repositories
\begin{lstlisting}
git remote add origin git@gitserver.com:owner/Project-name.git
git push -u origin master
\end{lstlisting}

\end{description}

\todo{example of git repo init and tex dir structure and something else.}

%
\subsection{Combining \LaTeX{} and Git}

%
Like all software, git works best if it is used for the purposes it was built
    for - ASCII text files, or in normal people language - pure text files.
%
Hence, anything, which can be actually be encoded in an ASCII file will be
    stored very efficiently in a git repository.
%
The examples of such files would be:
%
\begin{description}
    \item[Source Code Files] These are any files, which contain the source code
        for any particular program. This is usually the case as the compiler
        (software which produces executable files, in Windows known as
        \ftype{exe} files) can only read pure text files.

    \item[\ftype{tex} files] Yes, \TeX{} is written using ASCII file-formats and
        this is the reason which makes it highly portable across different
        operating systems.

    \item[\ftype{eps} or other vector graphics files] Yes, you \emph{can} store
        vector graphics in ASCII files, which makes it ideal for using with git.
        As a side effect, you will have the whole history of the file!

        \emph{NB} you need it to be truly vector graphics image, that means,
        that if you import a \ftype{jpg}, \ftype{png}, or any other raster
        graphics image into a \ftype{eps} figure, then it will definitely not
        work, as effectively you only change the extension of the file.

    \item[\ftype{csv} files] These files usually are used as a \emph{de facto}
        format to output experimental data.
\end{description}

%
As you see, if you are using \LaTeX{} typesetting system and you do not have to
    deal with raster images, then you can have a very efficient set of tools.

%
\subsection{Some thoughts on more sophisticated approaches}

\todo{remember what I wanted to tell here.}


% ----------------------------------------------------------------------
\section{Using other VCS solutions}
% ----------------------------------------------------------------------

Using other VCS solutions is possible and highly recommended for people, who
find GIT too hard. 
%
One very good alternative might be the Mercurial versioning system, which
receives a lot of praises amongst its users.
%
However, Mercurial seems to be not as popular as Git and hence, the resources on
the web might not be as elaborate.

For those people who know Subversion and CVS and claim that they are really good
alternatives, I would advise to look into Mercurial and Git very seriously and
consider switching over.
%
It is because the old SVN and CVS systems are slower, not as space-efficient and
they are not as flexible.

However, there is a SVN repository on the Chemistry Department servers, which
would help you very much in setting up VCS repository which would not be public.
%
If you do not mind spending time learning an obsolete technology, then please
start using SVN as soon as possible.
%
There are already good guides about how to use Subversion with \LaTeX{} and you
can find them on the
\uurl{https://secure.wikimedia.org/wikibooks/en/wiki/LaTeX/Collaborative_Writing_of_LaTeX_Documents\#The_Version_Control_System_Subversion}{\LaTeX{}
wikibook}.
%
Otherwise please ask the Computer Officers in the Department about how the
things are going towards a git repository on their servers.

\end{document}

% Editor configuration:
% vim: tw=80:spell:spelllang=en_gb


