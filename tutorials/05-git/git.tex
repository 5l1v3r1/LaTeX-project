% Clever thing to minimize the files I need to edit to change how the files are
% looking. The following one line comment is to trick VIM-LaTeX
\documentclass[
%    draft
]{scrartcl}
%documentclass{scrartcl}
\KOMAoptions{
    fontsize=10pt
}
\setkomafont{pagenumber}{\bfseries\upshape\oldstylenums}
\renewcommand{\titlefont}{\rm\bfseries\LARGE}
\usepackage{
    ifxetex, 
    ifdraft,
    ifthen
}

\ifxetex
  \usepackage{fontspec}
  \usepackage{xunicode}
  \defaultfontfeatures{Mapping=tex-text} % To support LaTeX quoting style
  \setromanfont{Gentium}
\else
  \usepackage[utf8]{inputenc}
  \usepackage[T1]{fontenc}
  \usepackage{lmodern,textcomp}
\fi

\usepackage{
%    standalone,
%    lastpage,
    geometry,
    scrpage2,
%    setspace,
    amsmath,
    caption,
    calc,
%    floatrow,
}
\usepackage{
%    xcolor,
    graphicx,
    tikz,
    chemfig
}
\usepackage[final]{listings}
\usepackage[version=3]{mhchem}
\usepackage[update,verbose=false]{epstopdf}
\usepackage[colorinlistoftodos,obeyDraft]{todonotes}
\usepackage{hyperref}


% Set the geometry
\geometry{
    paper = a4paper,
    top=3cm,
    bottom=4cm,
    footskip=1cm,
    marginparwidth=3.5cm,
    headsep=1cm
}
\ifoptiondraft{
\geometry{inner=1.5cm, outer=4cm}
}{
\geometry{inner=3.0cm, outer=2.5cm}
}

%\onehalfspacing

% Setup hyperref
\hypersetup{
    colorlinks,
    urlcolor=blue,
    breaklinks
}

\usetikzlibrary{
    arrows,
    decorations.pathmorphing,
    backgrounds,
    positioning,
    fit,
    petri
}

% Define collors
\definecolor{myyellow}{HTML}{FFFAC9}
\definecolor{myyellowl}{HTML}{FFFBDD}

% Define lstlisting env
\lstset{
    language=[LaTeX]TeX,
    backgroundcolor=\color{myyellowl},
    numbers=left,
    numberstyle=\footnotesize,
    breaklines=true,
    breakatwhitespace=true,
    print=true
}

% Renew 2 styles
\renewpagestyle{plain}{{}{}{}}{{}{}
{\hfill\pagemark{}}}
\renewpagestyle{headings}{{}{}{}}{{}{}
{\hfill\pagemark{}}}

% Set the headings page style
\pagestyle{headings}

% Text mode commands
\newcommand{\uurl}[2]{\href{#1}{#2}\footnote{The URL is \url{#1}}}
\newcommand{\ftype}[1]{\texttt{.#1}}
\newcommand{\fname}[2]{\texttt{#1.#2}}
\newcommand{\pkg}[1]{\texttt{#1}}
\newcommand{\env}[1]{\texttt{#1}}
\newcommand{\cmd}[1]{\texttt{\textbackslash{}#1}}
\newcommand{\usepkg}[2]{
    \texttt{\textbackslash{}usepackage%
    \ifthenelse{\equal{#2}{}}{}{[#2]}\{#1\}}}
\newcommand{\comment}[1]{}

% New commands which ease the work. Units and relative uncertainties
\newcommand{\unit}[1]{\ensuremath{\, \mathrm{#1}}}
\newcommand{\rel}[1]{\ensuremath{ \cfrac{\Delta #1}{#1}}}
\newcommand{\eten}[1]{\ensuremath{ \times 10^{#1}}}
\newcommand{\DP}[2]{\ensuremath{\cfrac{\partial #1}{\partial #2}}}
\newcommand{\DD}[2]{\ensuremath{\cfrac{\mathrm{d} #1}{\mathrm{d} #2}}}
\newcommand{\dd}[1]{\ensuremath{\mathrm{d}#1}}

% Alter some LaTeX defaults for better treatment of figures:
    % See p.105 of "TeX Unbound" for suggested values.
    % See pp. 199-200 of Lamport's "LaTeX" book for details.
    %   General parameters, for ALL pages:
    \renewcommand{\topfraction}{0.9}    % max fraction of floats at top
    \renewcommand{\bottomfraction}{0.8} % max fraction of floats at bottom
    %   Parameters for TEXT pages (not float pages):
    \setcounter{topnumber}{2}
    \setcounter{bottomnumber}{2}
    \setcounter{totalnumber}{4}     % 2 may work better
    \setcounter{dbltopnumber}{2}    % for 2-column pages
    \renewcommand{\dbltopfraction}{0.9} % fit big float above 2-col. text
    \renewcommand{\textfraction}{0.07}  % allow minimal text w. figs
    %   Parameters for FLOAT pages (not text pages):
    \renewcommand{\floatpagefraction}{0.7}      % require fuller float pages
    % N.B.: floatpagefraction MUST be less than topfraction !!
    \renewcommand{\dblfloatpagefraction}{0.7}   % require fuller float pages

\captionsetup{
    format          = plain,        %
    labelformat     = simple,       %
    labelsep        = period,       %
    justification   = default,      %
    font            = default,      %
    labelfont       = {bf,sf},      %
    textfont        = default,      %
    margin          = 0pt,          %
    indention       = 0pt,          %
    parindent       = 0pt,          %
    hangindent      = 0pt,          %
    singlelinecheck = false         %
}

\renewcommand{\thefigure}{\oldstylenums{\arabic{figure}}}

\setatomsep{5mm}
\setbondoffset{.5mm}
\setcrambond{2.5pt}{1pt}{2pt}
\setbondstyle{thick}
\renewcommand*\printatom[1]{{\footnotesize\ensuremath{\mathsf{#1}}}}


% Custom packages

\newcommand{\TODO}[1]{\todo[inline]{#1}}

\title{Using GIT Version Control System for Collaborative Work and Having
History of a Document}
\author{Ignas Anikevicius}

\begin{document}

\maketitle
\tableofcontents
\listoftodos{\vskip 1em}

%
\LaTeX{} is a very good tool for this purpose as the files are created in ASCII
    format and everybody can read it the same way.

% ----------------------------------------------------------------------
\section{What is version control system (VCS)?}
% ----------------------------------------------------------------------

%
There are many models of VCS, but the main idea in all of them remains the same.
%
This is the ability of registering the changes to the files. 
%
A typical workflow can be described as follows:
\begin{enumerate}
    \item Repository is initialized with initial files.
    \item Then a person copies the repository
    \item makes some changes to the files
    \item submits them back to the repository 
    \item clever algorithms detect  which part(s) of the files were altered 
    \item The files are updated and the changes being made are saved
    \item Go back to point 2.
\end{enumerate}

%
As you see from this description it is clear that the whole history of how the
    files were changing is saved and, thus, they can be restored to any previous
    state.
%
In addition to this being such a good back up tool, there are several
    other advantages one should be aware of:
%
\begin{itemize}
    \item It takes much less space than having multiple folders with different
        versions of the files
    \item One can spot what was changed much more easily
    \item People can work on different parts of the file at the same time.
\end{itemize}

%
The last feature is the most useful for \LaTeX{} users once they need to work
    together with someone.

% ----------------------------------------------------------------------
\section{Using GIT and \LaTeX{} together}
% ----------------------------------------------------------------------

%
\subsection{Introduction on Git}

%
Git is a fully open source Distributed Version Control System (DVCS) used for
    many major Open Source Software (OSS) projects.
%
This means, that you get an advanced tool totally for free and you have got an
    army of community members around the world from whom you can learn about git
    and find new 'tricks' almost every day.
%
Distributed VCS from a simple VCS differs in one aspect which makes the former
    much more flexible and simply better than the latter - every user has its
    own copy of the entire repository and can work on it in any possible way.
%
This has some implications:
\begin{itemize}
    \item The user can work off-line and then when he goes back on-line he can
        submit the number of changes to the main repository.
    \item If the main repository goes down or is compromised, you can get the
        source code back to its original state simply because every user has a
        copy of it.
\end{itemize}


%
What is more, git has introduced 'branching' mechanism into VCS.
%
Branching means, that you can create a branch of a repository and then make some
    changes on it and then merge it back whenever you want.
%
Or in the same way you can just abandon it, should you think that the changes you
    made where simply a mistake and you want the previous state of the
    repository back.
%
Or you can even 'cherry-pick' some changes from the branch and merge it back to
    the 'master' branch and discard the others.

%
This branching mechanism is very useful when a lot of people a working on the
    same set of files as every person may have one or several branches
    containing some changes and they can ask the manager of the project to merge
    those changes into the master branch.
%
The nature of git would also let everybody to monitor the changes in various
    branches, which makes everybody more aware of what is being changed and what
    is not.
%
What is more, one can discard individual commits, or group them, which makes the
    system even more flexible.

%
This way one can have a very powerful tool to manage all the changes to the
    files on the system.

%
\subsection{Installing and Using Git}

%
You can download the latest version of git by following
    \uurl{http://git-scm.com/download}{this link}.
%
Git system is very easy to use if you are used to working in shell (such as Bash
    or Zsh).
%
However, it might require more adjustment for those who have always been using
    Graphic User Interface and are not as comfortable without any buttons or
    icons to press.
%
Luckily there are GUI software, which might help you interface with a Git
    repository very easily:
%
\begin{description}
    \item[Linux]
        Linux GUIs are discussed on
            \uurl{http://stackoverflow.com/questions/1516720/git-gui-client-for-linux}{this}
            and
            \uurl{http://stackoverflow.com/questions/2141611/a-pretty-and-feature-rich-git-gui-for-linux}{this}
            page.
    \item[Mac] 
        There is a short discussion on Mac Git GUIs
            \uurl{http://stackoverflow.com/questions/83789/what-is-the-best-git-gui-on-osx}{here},
            but the answer would be probably that you should use either the
            default gitk or git gui, or download the gitx, which seems to be a
            separate gui for Mac.
    \item[Windows] 
        The best GUIs for windows are discussed on
            \uurl{http://kylecordes.com/2010/git-gui-client-windows}{this
            webpage}.
    \item[Github] 
        There is a github app for mac, which makes it very easy (no command line
            involved) to manage your github repositories. 
        %
        However, you need at least OS X 10.6 to use it and you need to register
            on \href{https://github.com}{github.com}.
        %
        Also, it will not manage any git repositories, which are not on the
            github.
\end{description}

%
\subsection{Essential git commands}

%
This section is probably more suited towards people who are using command line
    interfaces to git, but I think all people should be aware of basic git
    commands in order to understand the inner workings better.
%
There is a very good list of commands when you create a new 
\uurl{http://www.github.com}{GitHub} repository:
%
\begin{description}
    \item[Set up git] Commands for changing system wide settings:
\begin{lstlisting}
git config --global user.name "Your Name which will appear in log messages"
git config --global user.email yourname@email.com
\end{lstlisting}
        %
        The first command sets a username, which will be used for your all git
            operations, whereas the second command sets the email.

    \item[Initialize repository] Command for initializing repository:
\begin{lstlisting}
mkdir Project-name
cd Project-name
git init        # This is for initializing the git repository
\end{lstlisting}
        %
        The first command is a UNIX command for creating a directory, so if it
            is created already, you do not need to execute it.
        %
        The second command is for changing the directory, which also happens to
            be a UNIX way of doing it.
        %
        The third command tells git to initialize an empty repository.

    \item[Adding files] Commands for adding various files to already initialized
        repositories:
\begin{lstlisting}
touch README                    # Create an empty text file
git add README                  # Add file to the git repo file-list
git commit -m 'first commit'    # Commit the changes
\end{lstlisting}
        %
        The first command in this list creates an empty text file called README.
        %
        Like other file operations not involving git, it will work only on Mac
            OS X and Linux OS.
        %
        The second command adds the file into the list of tracked files, which
            essentially means, that the file is added to the staging area of the
            repository.
        %
        The changes will not be made until the third command is executed which
            contains a message summarizing the changes.

    \item[Synchronizing] Adding a remote server and synchronizing the
        repositories
\begin{lstlisting}
git remote add origin git@gitserver.com:owner/Project-name.git
git push -u origin master
\end{lstlisting}
        %
        The first command adds a remote server with which the local repository
            will be synchronized.
        %
        There can be several remote repositories, which would have different
            assigned names (e.g. origin, origin-back, origin-devel, etc.).
        %
        The last command is for uploading the changes to the remote server under
            a branch named master.
        %
        If you have several branches on your local machine, then you would
            basically need to specify a different branch instead of master (e.g.
            ia277-test, ia277-master, etc.).

\end{description}

%
\subsection{Combining \LaTeX{} and Git}

%
Like all software, git works best if it is used for the purposes it was built
    for - ASCII text files, or in normal people language - pure text files.
%
Hence, anything, which can be actually be encoded in an ASCII file will be
    stored very efficiently in a git repository.
%
The examples of such files would be:
%
\begin{description}
    \item[Source Code Files] These are any files, which contain the source code
        for any particular program. This is usually the case as the compiler
        (software which produces executable files, in Windows known as
        \ftype{exe} files) can only read pure text files.

    \item[\ftype{tex} files] Yes, \TeX{} is written using ASCII file-formats and
        this is the reason which makes it highly portable across different
        operating systems.

    \item[\ftype{eps} or other vector graphics files] Yes, you \emph{can} store
        vector graphics in ASCII files, which makes it ideal for using with git.
        As a side effect, you will have the whole history of the file!

        \emph{NB} you need it to be truly vector graphics image, that means,
        that if you import a \ftype{jpg}, \ftype{png}, or any other raster
        graphics image into a \ftype{eps} figure, then it will definitely not
        work, as effectively you only change the extension of the file.

    \item[\ftype{csv} files] These files usually are used as a \emph{de facto}
        format to output experimental data.
\end{description}

%
As you see, if you are using \LaTeX{} typesetting system and you do not have to
    deal with raster images, then you can have a very efficient set of tools.

%
\subsection{Example 1: Starting a \LaTeX{} document and a git repository}

%
An example how to get all the files of some LaTeX project might be the best
    way to learn how to make a git repository.
%
At first we can create a directory with structure as bellow:
%
\begin{lstlisting}
-- 2011-01-Some-Paper-Name/
|---------------------------------------------------------------------
|   Selecting a meaningful name might ease work for yourself and
|   others. In this example we have yyyy - mm - name format of the
|   name, which might be useful when arranging or zipping the folder
|   as it is clear what is inside the folder.
|---------------------------------------------------------------------
    |-- figs_ai/        % efficient format
    |-- figs_eps/       % efficient format
    |-- figs_pdf/       % efficient format
    |-- figs_jpeg/      % not-so-efficient format
    |-- figs_png/       % not-so-efficient format
    |-- figs_svg/       % efficient format
    |-- figs_tex/       % efficient format
    |-- figs_tiff/      % not-so-efficient format
|---------------------------------------------------------------------
|   It is very important to keep separate types of figures in
|   different folders as it might ease addition/exclusion of the files
|   when it comes to managing the repository.
|
|   Only vector graphics can be backed-up with plain git, however,
|   there are modules to do that for binary files or other non-ASCII
|   formated files (such as png,jpeg,tiff...)
|
|   If you convert a png into a pdf, or you import some raster
|   graphics files into Adobe Illustrator and then export it as a
|   vector format, it will not count as a vector graphics, as the
|   nature of the graphics does not change at all.
|
|   However, one might probably want to have raster graphics together
|   with other bits required to compile an article. Therefore one should
|   either limit himself with the number of raster graphics used in the
|   document to minimum or use only vector graphics.
|---------------------------------------------------------------------
    |-- refs/
        |-- references.bib  % This is the only needed file.
        |-- references.blg  % Not needed
        |-- references.bbl  % Needed only if BibTeX/BibLaTeX isn't used
|---------------------------------------------------------------------
|   The bibliography database, if kept in a .bib format, can be stored
|   in git as well, which means that one can have incremental file
|   change history of the whole database, which is a very nice side
|   effect.
|
|   However, you should not keep the log files (.blg) or the bbl file if
|   BibTeX/BibLaTeX is used as the .bbl file is created automatically.
|   However, if you do not use these things, then .bbl file would keep
|   all your bibliography and you *need* to keep it in the repository.
|---------------------------------------------------------------------
    |-- paper1.aux      % File created on the fly, not needed
    |-- paper1.log      % File created on the fly, not needed
    |-- paper1.out      % File created on the fly, not needed
    |-- paper1.pdf      % File created on the fly, not needed
    |-- paper1.tex      % Main source file!
|---------------------------------------------------------------------
|   Only the storage of the .tex file is useful in a git repository as 
|   other files are generated on the fly and are not needed.
|---------------------------------------------------------------------
\end{lstlisting}

%
The best way to exclude files from the repository is to use the
    \fname{}{gitignore} file (its name might be different on Windows OS).
%
To exclude unnecessary filetypes, just put these lines in to your
    \fname{}{gitignore} file:

%
\begin{lstlisting}[language=Bash]
# Exlude additional files created by tex.
# Please add more if you know the names of them.
*.aux
*.log
*.out
*.run.xml
*.toc
\end{lstlisting}
%
However, rather than excluding a lot of files, one can exclude \emph{all} files
    and then change the permissions so that some of the files are included after
    all. It can be done as follows:
%
\begin{lstlisting}[language=Bash]
# Exclude everything
*

# Revoke the above for the following:
!figs_*/ # although it might contain raster graphics, we want to have
         # them all in the repo
!*.tex   # Include .tex files
!*.pdf   # include .pdf files
\end{lstlisting}

%
Since we have all the file rules set up the only left thing is to add files to
    the repository it self.
%
This can be done via the following commands:
%
\begin{lstlisting}[language=Bash]
# Start tracking all files in the repository:
git add ./*
# Commit the addition, so the files would actually appear in the repo:
git commit -m "Initial Commit"
# Add a remote server
git add remote ia277@some.server.com
# Upload the files.
git push origin
\end{lstlisting}
%
Then you can make a branch, switch to it and then upload your branch as well.
%
If you want to know more information on how to do this, please refer to the
    all-might \href{http://www.google.co.uk}{google} or the
    \uurl{http://book.git-scm.com/}{git book}.

%
\subsection{Example 2: Downloading a git repository}

%
Downloading of the whole repository is more than simple.
%
Just open a terminal and then execute the following in the directory of your
    choice:
%
\begin{lstlisting}[language=Bash]
git clone git://some.host.ac.uk/the/address/of/the/repo.git
\end{lstlisting}

%
And in the existing directory a new directory with the repository name will be
    created with all the history of the repository.
%
If you need some assistance, please refer to the
    \uurl{http://book.git-scm.com/}{git book}.

%
\todo{When I have time talk about rsync with git and git bup or something
similar to get bin and ASCII files together}

% ----------------------------------------------------------------------
\section{Using other VCS solutions}
% ----------------------------------------------------------------------

Using other VCS solutions is possible and highly recommended for people, who
find GIT too hard. 
%
One very good alternative might be the Mercurial versioning system, which
receives a lot of praises amongst its users.
%
However, Mercurial seems to be not as popular as Git and hence, the resources on
the web might not be as elaborate.

For those people who know Subversion and CVS and claim that they are really good
alternatives, I would advise to look into Mercurial and Git very seriously and
consider switching over.
%
It is because the old SVN and CVS systems are slower, not as space-efficient and
they are not as flexible.

However, there is a SVN repository on the Chemistry Department servers, which
would help you very much in setting up VCS repository which would not be public.
%
If you do not mind spending time learning an obsolete technology, then please
start using SVN as soon as possible.
%
There are already good guides about how to use Subversion with \LaTeX{} and you
can find them on the
\uurl{https://secure.wikimedia.org/wikibooks/en/wiki/LaTeX/Collaborative_Writing_of_LaTeX_Documents\#The_Version_Control_System_Subversion}{\LaTeX{}
wikibook}.
%
Otherwise please ask the Computer Officers in the Department about how the
things are going towards a git repository on their servers.

\end{document}

% Editor configuration:
% vim: tw=80:spell:spelllang=en_gb


