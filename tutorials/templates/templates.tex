% Clever thing to minimize the files I need to edit to change how the files are
% looking. The following one line comment is to trick VIM-LaTeX
% \documentclass{article}
%documentclass{scrartcl}
\KOMAoptions{
    fontsize=10pt
}
\setkomafont{pagenumber}{\bfseries\upshape\oldstylenums}
\renewcommand{\titlefont}{\rm\bfseries\LARGE}
\usepackage{
    ifxetex, 
    ifdraft,
    ifthen
}

\ifxetex
  \usepackage{fontspec}
  \usepackage{xunicode}
  \defaultfontfeatures{Mapping=tex-text} % To support LaTeX quoting style
  \setromanfont{Gentium}
\else
  \usepackage[utf8]{inputenc}
  \usepackage[T1]{fontenc}
  \usepackage{lmodern,textcomp}
\fi

\usepackage{
%    standalone,
%    lastpage,
    geometry,
    scrpage2,
%    setspace,
    amsmath,
    caption,
    calc,
%    floatrow,
}
\usepackage{
%    xcolor,
    graphicx,
    tikz,
    chemfig
}
\usepackage[final]{listings}
\usepackage[version=3]{mhchem}
\usepackage[update,verbose=false]{epstopdf}
\usepackage[colorinlistoftodos,obeyDraft]{todonotes}
\usepackage{hyperref}


% Set the geometry
\geometry{
    paper = a4paper,
    top=3cm,
    bottom=4cm,
    footskip=1cm,
    marginparwidth=3.5cm,
    headsep=1cm
}
\ifoptiondraft{
\geometry{inner=1.5cm, outer=4cm}
}{
\geometry{inner=3.0cm, outer=2.5cm}
}

%\onehalfspacing

% Setup hyperref
\hypersetup{
    colorlinks,
    urlcolor=blue,
    breaklinks
}

\usetikzlibrary{
    arrows,
    decorations.pathmorphing,
    backgrounds,
    positioning,
    fit,
    petri
}

% Define collors
\definecolor{myyellow}{HTML}{FFFAC9}
\definecolor{myyellowl}{HTML}{FFFBDD}

% Define lstlisting env
\lstset{
    language=[LaTeX]TeX,
    backgroundcolor=\color{myyellowl},
    numbers=left,
    numberstyle=\footnotesize,
    breaklines=true,
    breakatwhitespace=true,
    print=true
}

% Renew 2 styles
\renewpagestyle{plain}{{}{}{}}{{}{}
{\hfill\pagemark{}}}
\renewpagestyle{headings}{{}{}{}}{{}{}
{\hfill\pagemark{}}}

% Set the headings page style
\pagestyle{headings}

% Text mode commands
\newcommand{\uurl}[2]{\href{#1}{#2}\footnote{The URL is \url{#1}}}
\newcommand{\ftype}[1]{\texttt{.#1}}
\newcommand{\fname}[2]{\texttt{#1.#2}}
\newcommand{\pkg}[1]{\texttt{#1}}
\newcommand{\env}[1]{\texttt{#1}}
\newcommand{\cmd}[1]{\texttt{\textbackslash{}#1}}
\newcommand{\usepkg}[2]{
    \texttt{\textbackslash{}usepackage%
    \ifthenelse{\equal{#2}{}}{}{[#2]}\{#1\}}}
\newcommand{\comment}[1]{}

% New commands which ease the work. Units and relative uncertainties
\newcommand{\unit}[1]{\ensuremath{\, \mathrm{#1}}}
\newcommand{\rel}[1]{\ensuremath{ \cfrac{\Delta #1}{#1}}}
\newcommand{\eten}[1]{\ensuremath{ \times 10^{#1}}}
\newcommand{\DP}[2]{\ensuremath{\cfrac{\partial #1}{\partial #2}}}
\newcommand{\DD}[2]{\ensuremath{\cfrac{\mathrm{d} #1}{\mathrm{d} #2}}}
\newcommand{\dd}[1]{\ensuremath{\mathrm{d}#1}}

% Alter some LaTeX defaults for better treatment of figures:
    % See p.105 of "TeX Unbound" for suggested values.
    % See pp. 199-200 of Lamport's "LaTeX" book for details.
    %   General parameters, for ALL pages:
    \renewcommand{\topfraction}{0.9}    % max fraction of floats at top
    \renewcommand{\bottomfraction}{0.8} % max fraction of floats at bottom
    %   Parameters for TEXT pages (not float pages):
    \setcounter{topnumber}{2}
    \setcounter{bottomnumber}{2}
    \setcounter{totalnumber}{4}     % 2 may work better
    \setcounter{dbltopnumber}{2}    % for 2-column pages
    \renewcommand{\dbltopfraction}{0.9} % fit big float above 2-col. text
    \renewcommand{\textfraction}{0.07}  % allow minimal text w. figs
    %   Parameters for FLOAT pages (not text pages):
    \renewcommand{\floatpagefraction}{0.7}      % require fuller float pages
    % N.B.: floatpagefraction MUST be less than topfraction !!
    \renewcommand{\dblfloatpagefraction}{0.7}   % require fuller float pages

\captionsetup{
    format          = plain,        %
    labelformat     = simple,       %
    labelsep        = period,       %
    justification   = default,      %
    font            = default,      %
    labelfont       = {bf,sf},      %
    textfont        = default,      %
    margin          = 0pt,          %
    indention       = 0pt,          %
    parindent       = 0pt,          %
    hangindent      = 0pt,          %
    singlelinecheck = false         %
}

\renewcommand{\thefigure}{\oldstylenums{\arabic{figure}}}

\setatomsep{5mm}
\setbondoffset{.5mm}
\setcrambond{2.5pt}{1pt}{2pt}
\setbondstyle{thick}
\renewcommand*\printatom[1]{{\footnotesize\ensuremath{\mathsf{#1}}}}


% Custom packages

\title{Using templates in this repository}
\author{Ignas Anikevicius}

\begin{document}

\maketitle

This document contains some instructions on how to use the templates hosted on
this website. 
%
On the other hand, most of the templates should be documented well enough so
that you could use them without any additional directions.
%
Therefore, this document will be more like a description of the templates.

The contents of the repository can be seen from the Table of Contents.
%
Just please refer to the appropriate section.


\tableofcontents
\clearpage

\section{Notes on the Usage of Journal Templates}

\subsection{The American Chemistry Society journals}

All the ACS journals are covered with the \verb|achemso| package. 
%
However, you should note, that it is not created by the ACS themselves, so all
queries should be sent to the author of the package directly and not to the
support email address found on the ACS homepage.
%
The files with the template can be found on the Chemistry Department
\LaTeX{} website and they are very well documented.
%
There is also achemso package
\uurl{http://mirrors.ctan.org/macros/latex/contrib/achemso/achemso.pdf}{documentation},
which can be found on the
\uurl{http://www.ctan.org/tex-archive/macros/latex/contrib/achemso}{CTAN package
directory}.
%
Should you have any questions about the usage of the package, this is probably
the best place to find your answers.

\subsection{The elsevier \LaTeX\ class}

This publishing company is also very kind to provide authors with their own
\LaTeX{} templates, which should do the job very well.

\subsection{Nature Publishing Group journals}

Nature previously was known to not support \LaTeX{} at all, but recently they
started accepting \verb|.tex| files.
%
Although they do not provide you with any templates, they accept \emph{any}
document, which can be compiled using one \emph{standard} document classes (such
as RevTeX, article, scrartcl, achemso).
%
As far as I looked into it, they are using \LaTeX{} themselves and they will
force their in-house style on your document, by making some adjustments.



There is a \LaTeX{} template on this website for NPG journals, which was created
by me and I tried to get the style as close as possible to the original.
%
What is more, I tried to make it compliant with standard document classes as
well, so that it would not be too hard to submit your paper.
%
However, I do not know whether they will like my template, so it should be usage
more for the review and preparation of the manuscript.

\subsection{PNAS Template}

The PNAS \LaTeX{} template is also good, but somehow, there were some issues
with the font handling (at least for me), so if you have any problems, you
\emph{should} turn off their font support at first, before searching any other
possible causes.
%
I have already commented out this command in the supplied templates on this
website.

\subsection{RSC \LaTeX\ Template}

RSC publisher \emph{can} accept \LaTeX{} files and they also provide all types
of templates you might need.
%
The templates are very well documented, and shouldn't cause any problems.

\subsection{Science \LaTeX\ Template}

Science publishers are not very \LaTeX{} friendly, but they still can accept
your \LaTeX{} typeset manuscripts.
%
However, there are several limitations, as to what packages you can use and how
you should typeset several things.
%
These limitations are thoroughly described
\uurl{http://www.sciencemag.org/site/feature/contribinfo/prep/TeX_help/index.xhtml}{on
their website}.
%
The template for their articles can also be found on the very same web page or
on the repository on the Chemistry Department \LaTeX{} website.

\subsection{Wiley-VCH Templates}

This publisher does not accept \LaTeX{} typeset journal articles, nor it is easy
to produce a \verb|.pdf| while using \LaTeX{}.
%
Nevertheless, it was a challenge for me to make templates for some journals
which would comply with general guidelines of \LaTeX{} typesetting system.
%
You can evaluate my attempt by trying the templates by yourselves, they are
hosted on this website and you'll have to put the \verb|.cls| file in your
working directory in order to use it properly.
%
However, probably a better choice would be to put it together with all other
\LaTeX{} class files so that you do not need to copy it to a new directory
every time you start writing a new article.

This package at the moment provides templates for these journals:
\begin{itemize}
    \item Angew. Chem.
    \item Angew. Chem. Int. Ed.
    \item Chem. Eur. J.
    \item Chem. Asian J.
    \item Small
    \item Adv. Mat.
    \item Adv. Func. Mat.
\end{itemize}

Here are the journals for which I plan to add support in the nearest future:
\begin{itemize}
    \item ChemBioChem
    \item ChemPhysChem
    \item J. Polym. Sci. Parts A
    \item J. Polym. Sci. Parts B
    \item Eur. J. Org. Chem.
    \item Macromol. Chem. Phys.
    \item Macromol. Rapid. Com.
    \item Macromol. Symp.
\end{itemize}

The template file is well documented enough to understand all the additional
commands, which are provided by the class file.


\section{Notes on the Usage of Other Templates}

\subsection{Thesis/Report template usage}

This template was created during the summer project and should comply all the
regulations which where got from
\uurl{http://www.admin.cam.ac.uk/offices/gradstud/exams/submission/}{this
website}.
%
If you spot any inconsistency in the formatting done using the template and the
formatting required by the examiners, then please contact me and I will fix it
ASAP.

The templates themselves consist of exemplary directory structure, so that your
working directory would be as tidy as possible which will definitely pay-off in
a long run.
%
You \emph{should} create more folders for any additional chapter and read
through the notes in the master \verb|.tex| file called \verb|head.tex|.

There is also an additional \verb|stats.tex| file for getting the statistics
from your thesis or report.
%
It scans the \verb|head.tex| file and composes a simple document automatically
and you can easily find word-count and other useful statistics.

\paragraph{Note!} This is known to work on \emph{Linux} OS and should work on
\emph{Mac} as they share the same \LaTeX{} distribution.
%
It was not tested on \emph{Windows} yet and I would be very grateful if somebody
with sufficient \LaTeX{} knowledge could test it under this OS.
%
This said, I can not be held responsible for any discrepancies from the actual
statistics of the file as this code was not written by me.
%
Also, you should check the documentation of the script if you have any concerns
or would like to know how to ensure the best results.
%
The documentation can be found on
\uurl{http://app.uio.no/ifi/texcount/documentation.html}{this website}.

\subsection{Curriculum Vitae}

\subsection{Letters}

\subsection{Posters}

\subsection{Presentations}


\end{document}

% Editor configuration:
% vim: tw=80:spell:spelllang=en_gb


